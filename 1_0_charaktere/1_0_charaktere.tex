\chapter{Spielercharaktere}
Die Charaktere, die in diesem Rollenspiel verkörpert werden können, unterscheiden sich meist erheblich. Jeder Charakter kann eine eigene Kombination von Klassen, Attributen, Fähigkeiten und Ausrüstungsgegenständen besitzen. Der Fluss und die Natur des Spiels können sich deshalb, je nach Zusammensetzung der Charaktere, von Session zu Session erheblich unterscheiden. Normalerweise verkörpern die Spieler jedoch eine Person, die, unter Einsatz ihrer Eigenschaften, im Spielverlauf ein bestimmtes Ziel erreichen will.
\section{Klassen}
Jeder Spielercharakter ist Teil einer der drei Klassen. Durch sie erhält der Charakter einen Bonus, der ihn regelmäßig in einer schwierigen Situation weiterhelfen kann.
\subsection*{Experte}
Diese \glqq Fachleute\grqq{} besitzen einzigartige Qualifikationen. Zu den Experten zählen Techniker, Wissenschaftler, Unternehmer, Capitaine, Trickbetrüger, Doktoren und alle anderen Professionen, die eine schwierige Ausbildung oder ein hohes Maß an Talent benötigen.\\
Bonus: Einmal alle 24 Stunden kannst du einen Check, bevor du erfährst ob er gelingt, erneut würfeln und das Ergebnis zwischen den beiden Würfen wählen.
\subsection*{Kämpfer}
Solche Individuen haben ein einzigartiges Talent dafür, ihrem Ziel unnachgiebig schwere Treffer zuzufügen. Diese Begabung kann durch Elite-Kampftrainings, natürliche Veranlagung, widrige Umstände oder gar Psychosen entstehen.\\
Bonus: Einmal alle 24 Stunden kannst du mit deiner Waffe bei einem Angriff maximale Schaden verursachen oder sie automatisch das Ziel treffen lassen.
\subsection*{Überlebenskünstler}
So manch einem gelingt es immer wieder, entweder durch Können oder durch pures Glück, den tödlichsten Gefahren auszuweichen und sich aus den schwierigsten Situationen unbeschadet herauszuwinden.\\
Bonus: Einmal alle 24 Stunden kannst du einen Angriff oder Effekt negieren, von dem du Verletzungen oder Schaden erleiden würdest.\\

Die Klasse stellt immer nur die erste, grobe Ausrichtung dar, die die Fähigkeiten eines Charakters annehmen können. Jeder Spielercharakter kann frei aus allen \glqq Bauteilen\grqq{} zusammengesetzt werden, und muss nicht fest in einen Spielstil fallen. Es kann aber helfen, einem Charakter eine Bezeichnung zu geben, die den Mitspielern eine Vorstellung davon gibt, wozu er in der Lage ist. Man könnte seinen Charakter zum Beispiel als \glqq Unterhändler\grqq{}, \glqq Plasmawissenschaftler\grqq{}, \glqq Cowboy\grqq{}, \glqq Dieb\grqq{} oder \glqq Gladiator\grqq{} bezeichnen, solange man ihn dann auch geschickt zusammensetzt. Gerade ungewöhnliche Kombinationen können überraschend effektiv sein!
\section{Charakter-Attribute}
Die körperliche und geistige Beschaffenheit eines Charakters wird in seinen Attributen festgelegt. Je höher eins dieser sechs Attribute ist, desto stärker ist diese Eigenschaft generell bei einem Charakter ausgeprägt.
Dies sind die Charakter-Attribute:
\begin{itemize}
\itemsep0pt
\item Stärke (STÄ)
\item Geschicklichkeit (GES)
\item Konstitution (KON)
\item Intelligenz (INT)
\item Weisheit (WEI)
\item Charisma (CHA)
\end{itemize}
Ein Charakter-Attribut gewährt einem Charakter einen Bonus auf den zugehörigen Attributs-Check. Dieser Bonus ist gleich der Anzahl der Punkte, die der Charakter in diesem Attribut besitzt (beispielsweise hat ein Charakter mit einer Stärke von 2 bei einem Stärke-Check einen Bonus von +2). In ein einzelnes Attribut können nicht mehr als insgesamt 5 Punkte investiert werden.

Ein Charakter auf Level Null verteilt bei seiner Erstellung einmal +2, zwei mal +1 und einmal -1 auf seine Attribute.

Jedes Mal, wenn ein Charakter ein Level aufsteigt, kann er einen weiteren Attributs-Punkt verteilen.
\section{Geschwindigkeit}
Die Geschwindigkeit eines Charakters entscheidet darüber, wie weit er sich an seinem Zug bewegen kann. Jeder Charakter hat eine Grundgeschwindigkeit von 4 Längen, deren Gesamtreichweite er während seines Zuges beliebig auf die Zeit vor und nach seiner Aktion verteilen kann (eine Länge entspricht 5 Metern). Bestimmte Fähigkeiten und Ausrüstungen können die Geschwindigkeit eines Charakters verändern.
Ein Charakter kann seine Aktion zum Sprinten verwenden, und damit in einem Zug weitere 4 Längen zurücklegen.
\section{Charakter-Fähigkeiten}
Jeder Charakter erhält bei seiner Erstellung drei Fähigkeiten, die ihm Vorteil bei einem entsprechenden Attributs-Check oder eine andere nützliche Eigenschaft verleihen. Gleichzeitig stellen diese drei Fähigkeiten seine Trefferpunkte dar. Wird ein Charakter bei einem gefährlichen Check oder von einem Angriff verletzt, so muss er eine dieser Fähigkeiten aussuchen und wegstreichen. Ist eine Fähigkeit weggestrichen, so kann sie ab diesem Zeitpunkt nicht mehr genutzt werden. Hat ein Charakter keine Fähigkeiten mehr, so wird er ausgeschaltet. Ein Charakter erhält alle seine Fähigkeiten am Ende einer Pause zurück.
Fähigkeiten werden unterteilt in Grundfähigkeiten und Spezialfähigkeiten. Manche Fähigkeiten sind, der Übersichtbarkeit halber, in thematisch zueinander passende Unterkategorien eingeordnet.
Erreicht ein Charakter das fünfte Level, so erhält er eine vierte Fähigkeit, die er sich von der Liste aller Fähigkeiten aussuchen kann.
Jede Fähigkeit kann bei einem Charakter nur einmal gewählt werden.
\todo{Fähigkeiten einfügen}
\section{Inventar-Management}
Jeder Charakter kann eine begrenzte Anzahl an Objekten an seiner Person führen. Jeder Gegenstand hat ein Gewicht und ein Charakter kann das addierte Gewicht von Gegenständen nur transportieren, wenn es nicht größer als seine Tragelast ist, oder er Hilfsmittel wie Transportfahrzeuge oder einen Rucksack zur Verfügung hat.

Die Tragelast eines Charakters, der seinen Einsatz-Anzug trägt, ist 6.

Die Tragelast einer Person ohne solche Hilfsmittel ist 5.

Ein Charakter kann ein Objekt mit einem Gewicht von 1 in der Regel verborgen halten, ohne dass es von außen sofort erkannt wird. Dafür kann er einen Tarnungs-Check durchführen, um den Gegenstand zu verstecken. Diese Tarnung kann mit einem Weisheit(Wahrnehmung)-Check gegen den Wert des Tarnungs-Checks durchschaut werden.

Für jeden Punkt der Tragelast deines Charakters, die freigelassen bleibt, erhält er auf Nahkampf Abwehren- und Fernkampf Ausweichen-Checks einen Bonus von +1, bis zu einem Maximum von +3.
Ein Charakter kann immer einen Gegenstand ausgerüstet haben, und während seines Zuges einmal zu einem anderen Gegenstand wechseln.
Jeder Charakter hat in seinem Inventar ein Handy, das keinen Einfluss auf die Tragelast hat. Zudem kann jeder Charakter ein paar leichte Gegenstände mit sich führen, die zusammen in eine Hand passen müssten. Diese dienen dazu, dem Spieler mehr Möglichkeiten beim Darstellen seines Charakters zu geben.

Beispiele hierfür wären ein Notizblock, eine Taschenuhr, ein Glücksbringer oder ein Flachmann.
\section{Einsatz-Anzüge}
Was könnte praktischer sein, als seine wichtige Ausrüstung ständig wortwörtlich am Körper zu tragen? Charaktere nutzen, egal auf welchem Gebiet, meist einen speziell auf sie zugeschnittenen Einsatz-Anzug. Dieser enthält standardmäßig Kommunikationssysteme, eingewobene Kunstmuskel-Filamente, welche die Tragelast erhöhen, und dient als leichte Körperpanzerung. Die Anzüge sind weit verbreitet, zum Beispiel bei physisch fordernden Arbeiten, Militäreinheiten, Elite-Firmenwissenschaftlern und reichen Verbrecherorganisationen.

Der einzige wirkliche Nachteil an der ganzen Sache ist, dass diese Anzüge recht klobig sind und ihre Träger deswegen schon aus einiger Entfernung als potentielle Prioritätsziele erkannt werden können. Die meisten Einsatz-Anzüge bieten keinen anhaltenden Schutz gegen Gefahren wie den freien Weltraum, radioaktive Verstrahlung, giftige Atmosphäre, Wassereinbruch, usw. Hierfür müssten jeweils passende Schutzanzüge angelegt werden.

Manche Charaktere leben fast dauerhaft in ihren Anzügen und verlassen sie nur zur Wartung, andere macht nichts glücklicher, als nach einer erfolgreichen Mission wieder in \glqq leichtere\grqq{} Kleidung schlüpfen zu können. Ein Anzug kann mit einer Aufladung für 3 Tage aktiv bleiben, danach wird er für den tragenden Charakter eher zu Last. Ist die Batterie leer, so bietet der Anzug keine Boni mehr.

Optionen für die Anzüge sind:
\begin{itemize}
\itemsep0pt
\item Spezialisierungs-Anzug: Erhöhe ein Charakter-Attribut um 1, solange dieser Anzug aktiv ist.
\item Schutzanzug: Dieser Anzug bietet Schutz gegenüber einer lebensgefährlichen Umgebung.
\item Dynamo-Anzug: Solange ein Charakter diesen Anzug trägt, wird seine Batterie nicht erschöpft.
\item Last-Anzug: Erhöhe die Tragelast deines Charakters erneut um +1.
\item Tarn-Anzug: Du erhältst einen Bonus von +3 auf Geschicklichkeit(Tarnung)-Checks.
\item Kontroll-Anzug: Du erhältst einen Bonus von +2 auf alle Checks, mit denen du ein Fahrzeug kontrollierst.
\item Mobility-Anzug: Erhöhe die Geschwindigkeit deines Charakters um 2 Längen.
\end{itemize}

\section{Hintergrundgeschichte}
Erstellt ein Spieler seinen Charakter, so sollte er nicht nur notieren, was seine Fähigkeiten sind. Genauso wichtig ist es, sich zu überlegen, WIE und WARUM ein Charakter diese Fähigkeiten dann einsetzt. Ein Spieler darf seinen Charakter nicht einfach nur den Bösewicht erschießen lassen, es muss ihm dazu auch noch ein cooler Spruch einfallen!

Ein Charakter erhält durch seine Hintergrundgeschichte eine Hintergrundfähigkeit, zusätzlich zu seinen drei anderen Fähigkeiten. Dieses dient nicht dazu, die Trefferpunkte eines Charakters zu messen, und kann auch nicht weggestrichen werden.
\subsubsection{Persönlichkeit}
Ein Spieler sollte in diesem Feld notieren, wie sich sein Charakter in bestimmten Situationen oder im Umgang mit anderen generell verhält.
\subsubsection{Ideale und Wertvorstellungen}
Hier werden die Personen, Orte, Gegenstände und Konzepte vermerkt, die einen besonders großen Einfluss auf den Charakter haben oder hatten.
\subsubsection{Schwächen}
Niemand ist perfekt, und erst durch ein paar Schwächen wird ein Charakter wirklich interessant. Solche Charaktere bereichern das Spiel mehr, als dass sie ihm schaden.
\subsubsection{Herkunft von Charakteren}
Praktisch jeder Charakter lässt sich in eine dieser \glqq Bevölkerungsschichten\grqq{} einordnen. Spieler können ein Mitglied der Spitze der Gesellschaftspyramide verkörpern, oder ganz unten anfangen. Solange es Sinn macht, dass die Spielercharaktere zusammenarbeiten, ist alles erlaubt. Und mit etwas Fantasie lässt sich schon einiges weg-erklären!

\subsection{Bevölkerungsgruppen}
\subsubsection{Bürger}
Ein Klassiker: Bürger bilden den größten Anteil der Bevölkerung in quasi jedem mehr oder weniger zivilisierten Ort weltweit. Der größte Anteil der Bürger ist die Arbeiterklasse, nur ein kleiner Prozentsatz dieser Bevölkerung schafft es jemals, sozial aufzusteigen. Das daraus resultierende Einkommen wird dann fast immer dazu genutzt, der kläglichen Existenz zu entkommen. Schnell ist der monatliche Gehaltscheck in eine Urlaubsreise, ein Serienabonnement oder einen virtuellen Trip investiert.

Im Gegensatz dazu steht die Bevölkerung der ländlichen Communities, die oft ihre ursprüngliche Kultur und eine einfachere Lebensweise bevorzugen. Dies führt, besonders in abgelegenen Ortschaften mit wenig Durchgangsverkehr, zu einer mitunter rassistischen oder extremistischen Haltung. Aus solchen Gegenden stammende Personen können, wenn sie ihre Heimat verlassen, eine ganze Reihe von Vorurteilen und Schreckgeschichten nur schwer ablegen. Ein geringer Grad an Veränderungen, wie beispielsweise eine Erhöhung der Reaktionsgeschwindigkeit durch Implantate und ein VR-Anschluss, sind auch bei vielen Bürgern anzutreffen.
\subsubsection{Beamte}
Diese Leute, \glqq Beamte\grqq oder \glqq Arschkriecher\grqq genannt, werden fast nur in den Megastädten angetroffen. Oft nutzt eine Familie der gehobenen Mittel- oder niederen Oberschicht fast ihr gesamtes Kapital, um ihren Nachwuchs schon vom jüngsten Alter an mit verschiedensten Methoden aufzurüsten. Hormon-Behandlungen, Implantate und Operationen perfektionieren Beweglichkeit, Reflexe und Aussehen. Immer wieder besitzen ganze Generationen die Gesichter der beliebtesten Serien-Stars der letzten Saison. Schwächen in der reinen physischen Kraft, meist resultierend aus jahrelanger Schreibtischarbeit, werden durch eine gute Bildung und Talent in \glqq zwischenmenschlicher Problemlösung\grqq  ausgeglichen.

Viele Beamte betreiben sehr viel Aufwand, um reich und bekannt zu wirken, leben dabei aber über ihre Verhältnisse. Dies führt dazu, dass sie sowohl vom Adel, als auch von den einfacheren Arbeitern, verachtet werden.
\subsubsection{Adel}
Die höchsten Ehrentitel und der größte Reichtum, den man sich vorstellen kann: Die Adels-Familien, im Besitz der größten Firmen des Sonnensystems, und einige steinreiche Spezialindividuen, stellen die elitäre Oberschicht dar. Entstanden ist diese Schicht aus den CEOs der großen Unternehmen des 21. Jahrhunderts und aus Promis, Stars und Superreichen. Oft abwertend \glqq Bonzen\grqq genannt, führen sie meiste ein sorgloses Leben im Luxus, nur gefährdet vom Konflikt zwischen den mächtigeren Clans.

Mit den besten medizinischen Mitteln und schon als Embryos behandelt, prägen sie weltweit das Schönheits-Ideal und müssen sich gleichzeitig kaum Sorgen um Krankheit oder Alter machen. Die Kinder bekommen nur die besten Lehrer, müssen den höchsten Ansprüchen genügen und gleichzeitig wird ihnen eine absolute Hochnäsigkeit in die Wiege gelegt. Diese elitäre Umgebung verlassen nur die absolut (und manchmal enterbten) Eigensinnigen, Bewährungswillige und Mitglieder derjenigen Familien, die den Machtkampf und somit ihr Kapital verloren haben.

\textbf{Clans:} Die Adeligen unterteilen sich in Clans, bei denen sich verschiedenste Merkmale und Mentalitäten durchgesetzt haben. Meist sind eine oder mehrere Firmen oder Monopole fest mit einer bestimmten Familie assoziiert. Ständige Intrigen und Machtwechsel sorgen für ein sich immer wandelndes Umfeld, in dem man leicht unter die Räder gerät.
Die Erbreinfolge eines Clans ist oft nicht auf Blutsverwandtschaft basiert, da der Nachwuchs sowieso bis zur Unkenntlichkeit Genmanipuliert wird. Stattdessen dient die Erziehung und das Erfüllen von besonderen körperlichen und geistigen Voraussetzungen als \glqq Nachfolgekriterium\grqq. Manchmal werden Personen, die sich den Clananführern besonders beweisen konnten, sogar in die Familie aufgenommen.

\textbf{Gefallene Clans:} Ein gefallener Clan hat einen Großteil seiner Macht an einen anderen Clan verloren, aber das bedeutet noch nicht unbedingt das Ende. Viele dieser Familien begeben sich in die Unterwelt und nutzen die restlichen Ressourcen, um einen Schatten ihrer selbst aufrecht zu erhalten. Andere gründen kleine unabhängige Unternehmen und versuchen, der erdrückenden Konkurrenz der Megakonzerne zum Trotz, die soziale Leiter nicht weiter herabzurutschen.
\subsubsection{PersOhnen}
Slummer, Punks und Ökos. Eins haben sie alle gemeinsam: Diese Leute wollen oder haben keinen offiziellen FF-Perso Quantum. Dies kann verschiedene Gründe haben, die Folgen sind aber meistens die gleichen: eingeschränkte (wenn überhaupt) Menschenrechte, wenig Chancen auf eine Karriere oder sozialen Aufstieg und höchstwahrscheinlich ein Leben in einem dreckigen Slum oder einer zurückgebliebenen Gemeinde. Das Schicksal (und oft auch schon die Natur der nächsten Mahlzeit) bleibt ungewiss, die Existenz ist ein ständiger Kampf. Die meisten würden Alles für eine Chance geben, diesen Umständen zu entkommen.

\textbf{Slummer:} Dieser abwertende Begriff dient als Sammelbzeichnung für all jene armen Seelen, die in den unteren Schichten der Sprawls und in den Slums der Wert leben müssen. Oft findet in winzigen, überfüllten und von den ursprünglichen Erschaffern schon lange verlassenen Behausungen das Leben von ganzen Generationen statt. Großfamilien versuchen sich mit unterbezahlten oder illegalen Jobs über Wasser zu halten. Gleichzeitig hofft man, nicht den anderen gefährlicheren Bewohnern dieses Milieus zum Opfer zu fallen. Werkzeuge und Technologie werden von den Eltern an ihre Kinder weitergegeben und für Außenseiter auf oft unverständliche Weise repariert, umfunktioniert oder recycelt.

\textbf{Punks:} Der Begriff hat sich als Bezeichnung für die Vielzahl an Gangmitgliedern eingebürgert. Normalerweise quetschen sie durch ihre überlegene rohe Kraft oder Hinterlistigkeit aus ihren Mit-Slummern und unglücklichen Bürgern ihren Lebensunterhalt heraus. Anarchistische und Nihilistische Züge sind bei den Punks typisch, aber auch andere Ideologien finden bei ihnen Anklang. Manche genießen diesen Lebensstil vollauf, während andere in den \glqq Dienst\grqq{} gepresst oder von den Umständen gezwungen werden. Punks beanspruchen oft die beste Technologie ihres Reviers für sich, und richten sich festungsartige Heimatbasen (\glqq Keep\grqq genannt) ein, an die sich normale Polizeikräfte nicht heranwagen. Meist werden diese Gangs hierarchisch geleitet und liegen in ständigen Krieg mit einander und den für die Region zuständigen Firmen oder Regierungen.

\textbf{Ökos:} Die Ökos haben ihren Namen ursprünglich von einer politischen und spirituellen Bewegung erhalten, die der extremen Industrialisierung und High-Technologisierung der Welt Einhalt gebieten wollte. Das Ziel war es, die Menschheit ihren Ursprüngen wieder näher zu bringen. Für einige bedeutet dies eine Stamm-Basierte Lebensart ohne Internet und Smarte Technologie (Sprich: Ohne Kontrolle durch Firmen und Regierungen) einzuführen. Ökos bevorzugen es, außerhalb der Städte zu siedeln und legen viel Wert auf Selbstversorgung. Viele Stämme haben eine bestimmte Kultur angenommen und vollständig verinnerlicht, was oft zu Aberglauben und Spiritualität führt. Von ihren Eltern wird ihnen ein Hass auf Technologie und Firmen eingeprägt, der auch gewaltsam ausgelebt wird. Die Öko-Gemeinschaften stehen unter dem Generalverdacht, Brutstätten für Terroristen und Psychopathen zu sein, die in den Städten immer wieder Anschläge verüben.
\subsubsection{Raumler}
Als Raumler gelten diejenigen, die nicht im Gravitationsfeld eines Planeten auf die Welt gekommen sind, sondern auf Raumstationen, -schiffen und kleineren Himmelskörpern. Raumler sind eher selten, da biologische Fortpflanzung in der Schwerelosigkeit nicht unproblematisch ist und KIs normalerweise in großen, gut ausgestatten Laboren der Fabriken geschaffen werden (und diese finden sich überwiegend auf der Erde). Personen, die im All aufgewachsen sind, entwickeln einen fast tänzerischen Fortbewegungsstil unter Mikroschwerkraft und können sich auch ohne Sicherheitsleine elegant an der Außenseite einer Raumstation entlangschwingen.

Dafür leiden sie umso stärker, sollten sie doch einmal in das Schwerefeld eines Planeten geraten. Grobmotorik, Erschöpfung und Kreislaufprobleme sind alles häufige Symptome vom Raumlern, die längere Zeit außerhalb ihres \glqq natürlichen\grqq  Umfelds verbringen müssen. Dennoch zeigen sie meist großen Stolz bezüglich ihrer Herkunft (manche Adligen reisen extra in den Weltraum, nur um da ihre Kinder zu bekommen), schließlich erinnern die Planeten \glqq von dort Oben aus gesehen\grqq eher an bunte Murmeln. Raumler zeigen wegen dieser Haltung auch eine gewisse Weltfremdheit, was die irdische Politik und Kultur angeht, bis zum Punkt der Verschrobenheit.
\subsubsection{Hirne}
Hirne sind eine besonders vielseitige Version von Künstlicher Intelligenz (kurz KI), die Personenstatus erlangen kann. Sie sind praktisch nur ein Behältnis, in der sich ein Gehirn auf Computer- oder biotechnischer Basis befindet. Dieses ist das einzige wirklich unersetzbare Teil des \glqq Körpers\grqq  eines Hirns, umgeben von  Energieversorgungssystemen, Speicher- und Recheneinheiten, Schutzplatten, Gliedmaßen, Anschlüssen, Bildschirmen, usw. Hirne legen ihren Fokus jedoch meist auf ihre \glqq geistige\grqq  Leistungsfähigkeit, die sie ständig anpassen und ausbessern können. Ständige Upgrades können sich jedoch nur wenige Hirne leisten, ein Großteil muss in Serverfarm-ähnlichen Zuständen ständig monotone Arbeiten ausführen (in Datenbanken, Customer-Support, Simulationen, etc.).
Bestehende Hirne können nicht perfekt kopiert werden, da ihre Struktur extrem komplex und im ständigen Wandel ist, zudem werden sie von ihren Herstellern oft als Black Box konzipiert. Hirne aus der gleichen \glqq Produktreihe\grqq ähneln sich Oberflächlich stark, doch je länger sie existieren, desto mehr prägt sich eine individuelle Persönlichkeit aus. Um sich den Personenstatus zu verdienen und Menschenrechte zu erhalten, müssen Hirne einen von der jeweils lokalen Regierung durchgeführten Test absolvieren, wonach sie von Vertretern des Instanz-Rates nochmals kontrolliert werden. In manchen Gebieten sind diese Tests absichtlich schwer konzipiert, um die KIs als billige und gewerkschaftslose Arbeitskräfte nutzen zu können.
\subsubsection{Dienstleistungs-Droiden}
Diese Androiden und Gynoiden werden geschaffen, um sich nahtlos in eine menschengemachte Umgebung einzufügen. Moderne Technologie erlaubt es diesen Maschinen, das Aussehen und Verhalten von Menschen nahezu perfekt zu simulieren. Viele Hersteller bestehen darauf, ihre Produkte äußerlich durch kleine Details vom Menschen unterscheidbar zu machen, eine möglichst ästhetische Form ist aber das Endziel. Ein klassischer Einsatzbereich von diesen Droiden findet sich bei Adeligen und Beamten, wo sie aus Haushälter und \glqq Spielzeuge\grqq  dienen. Auch das \glqq Vorzeigepersonal\grqq  vieler Firmen und Freizeiteinrichtungen setzt sich aus diesen Droiden zusammen.

Der Großteil dieser Dienstleistungs-Roboter ist geistig so limitiert, dass er keinen Personenstatus erlangen kann, auch Lern- und Improvisationsfähigkeit sind stark einschränkt. Manchen Hausherren genügt solch ein simpler Diener nicht, deswegen werden auch Roboter mit höheren kognitiven Fähigkeiten geschaffen. Diese können dann aber auch, je nach Rechtslage, die Testes zum Erlangen des Personenstatus durchführen. Dies geschieht meistens, wenn die alten Besitzer auf irgendeine Weise wegfallen, er sich selbst \glqq befreit\grqq , \glqq befreit wird\grqq  oder einfach großzügiger Weise direkt als eigenständige Person geschaffen wurde.

Dienstleistungs-Droiden werden für soziale Interaktion geschaffen und sind gewöhnlicherweise darin auch unschlagbar. Dafür sind ihre Körper nicht für extreme Belastungen oder Kampfsituationen geschaffen, weshalb sie Gefahrenzonen eher meiden sollten.
\subsubsection{Unabhängige Arbeitsroboter}
Der menschliche Körper ist für die Arbeit in modernen Massenfabriken nicht geschaffen, und deshalb wird dieser Job inzwischen meist von Maschinen übernommen. Ohne die Limitationen eines humanoiden Körpers können diese Roboter perfekt an ihre Aufgabe angepasst werden, zum Beispiel mit einem widerstandsfähigen Metallgehäuse, austauschbaren Greif- und Werkzeugarmen, Schienensystemen oder Panzerketten, speziell zugeschnittenen Sensoren, usw. Die meisten dieser Roboter besitzen keine Intelligenz und folgen monoton den Anweisungen des Fabriksystems. Doch für einige komplexe oder gefährliche Arbeiten ist ein gewisses Maß an Autonomie und Intelligenz nötig, weshalb einige Arbeitsroboter mit leistungsfähigen Recheneinheiten ausgestattet werden. Und aus diesem \glqq Proto-Gehirn\grqq kann sich mit genügend Zeit ein eigener Verstand entwickeln, auch wenn sich dieser meist (für Menschen) erschreckend von dem Verhalten gewöhnlicher Kreaturen unterscheidet. Diese Roboter können ebenfalls einen Test zum Erlangen eines Personenstaus durchführen, sollten sie das denn überhaupt wollen. Sie glänzen dann durch ihre pure Stärke, Präzision und Ausdauer, auch außerhalb einer Fabrikhalle.
\subsubsection{Bioformte}
Als Bioformte (oder abwertend \glqq Freaks\grqq) werden Personen bezeichnet, die durch genetische Manipulation auf der Basis von anderen Lebewesen künstlich erzeugt wurden. Besonders beliebt war einige Zeit das \glqq Upliften\grqq, was den Versuch darstellt, Tiere auf den gleichen kognitiven Stand wie den Menschen zu bringen. Die ersten Erfolge wurden mit Menschenaffen erzielt, die durch Gehirnvergrößerung und andere Modifikationen zu vollwertigen Personen wurden (auch wenn die Erzeugung einer neuen Generation dieser \glqq Affenmenschen\grqq extrem kostspielig ist). Ähnliche Prozesse wurden an einer Vielzahl von anderen Kreaturen (meist die mit hoher natürlicher Intelligenz oder anderen begehrenswerten Eigenschaften) wie Delfinen, Krähenvögeln, Kopffüsslern und diversen Haustieren durchgeführt, oft mit gemischten Erfolgen. Vielerorts verpönt ist das einmischen von tierischem Erbgut in einen menschlichen Embryo, was zwar in \glqq funktionsfähigen\grqq  Personen resultieren kann, aber oft unberechenbare Nebenwirkungen hat. Für einige Zeit war es in Adelskreisen \glqq chic\grqq , Nachkommen mit den Ohren, Augen oder Gliedmaßen von Tieren auszustatten (alles im Sinne der Ästhetik). Die Kosten dafür sind aber oft genetische Instabilität, Sterilität, Autoimmunreaktionen und Mutationen.
Viele religiöse oder idealistische Organisationen und Staaten verbieten das Bioformen von Lebewesen und schränken die Menschenrechte von Bioformten stark ein. Dort werden sie manchmal sogar verfolgt und gelten als Monster oder Frevel gegen die Menschheit.

\subsection{Personalausweise}
In fast jedem Land und Firmenstaat ist das ständige mitführen eines Personalausweises Pflicht. Globaler Standard ist der FF-Perso Quantum (kurz: Perso), der von Friedl-Fuchi produziert wird. Ein FF-Perso Quantum ist ein fingernagelgroßer Chip mit fälschungssicherer Identifikationsnummer, über den sich eine Person weltweit an Chip-Lesegeräten ausweisen kann. Jemand ohne einen \glqq Perso\grqq fällt innerhalb eines von Polizei oder Sicherheitskräften bewachten Gebieten schnell auf, zumindest solange man sich in den \glqq rechtschaffenden\grqq Bezirken von Städten aufhält.

Der FF-Perso Quantum übermittelt dauerhaft Personendaten, Standort und vieles mehr an die Überwachungs- und Werbesysteme der Umgebung, was die meisten Firmen dazu nutzen, das Stadterlebnis zu \glqq personalisieren\grqq (sprich, Profit zu schlagen). Er enthält aber auch praktische Funktionen wie Zugang zu öffentlichen Verkehrsmitteln, Hotels, erleichterter Navigation und eine automatischen Alarmfunktion bei Überfällen.

Personengruppen, die ihre Machenschaften verborgen halten wollen, führen oft mindestens einen weiteren Perso mit sich, um im Flug auf eine andere Identität wechseln zu können. Normalerweise erkennen nur die ausgereiftesten Überwachungssysteme solch einen Wechsel in Echtzeit (oder überhaupt) und ein zweiter Perso ist für Verbrecher inzwischen oft genauso wichtig wie eine Gesichtsmaske. Dennoch ist das Führen von mehreren Persos, der Versuch der Duplikation des Identifikationscodes oder die unbeaufsichtigte Herstellung streng verboten, und wird gnadenlos verfolgt.
\section{Connections}
Ein Charakter hat für gewöhnlich ein kleines Netzwerk aus Personen, genannt Connections, die er gut kennt oder mit denen er schon einmal zusammengearbeitet hat. Dies können Familienmitglieder, Freunde, Händler, Spezialisten oder Dienstleister sein, mit denen der Charakter besonders gute Verbindungen hat.

Jede dieser Personen hat dem Charakter gegenüber einen Loyalitäts-Wert. Dieser symbolisiert die Willigkeit (oder den Zwang) der Connection, dem Charakter auszuhelfen und oft auch den Grad der Vertrautheit mit dem Charakter. Jeder Punkt in diesem Wert gibt dem Charakter einen Bonus von +1 auf die Loyalitäts-Checks, die er ausführen kann, um den Charakter dazu zu bringen, etwas für ihn zu tun (genauso kann negative Loyalität ein Malus auf einen solchen Check verursachen). Zum Beispiel kann ein Charakter einen altbekannten Waffenschmuggler mit einem hohen Loyalitäts-Wert leichter dazu bringen, eine schwere Schusswaffe in einem Hochsicherheitstrakt zu verstecken.

Um einen Loyalitäts-Check durchzuführen, würfelt der Spieler 1d20 + Charisma + Loyalitäts-Wert des jeweiligen Charakters. Der GM entscheidet die Schwierigkeitsklasse dieses Checks, je nach der Natur und/oder Absurdität des Anliegens.

Ein Charakter kann den Loyalitäts-Wert erhöhen oder eine neue Connection erlangen, wenn er etwas besonders Altruistisches für die betreffende Person tut. Hierfür müsste er dem Charakter beispielsweise das Leben retten oder eine einzigartige Geschäftschance gewähren.
Genauso kann ein Charakter den Loyalitäts-Wert eines Charakters senken, indem er ihn schlecht behandelt, unnötig in Gefahr bringt oder vernachlässigt.

Jeder Charakter erhält bei seiner Erstellung zwei Connections mit einem Loyalitäts-Wert von 1.
\section{Karma-Punkte}
Jedes Mal, wenn ein Charakter ein Level aufsteigt, erhält er einen Karma-Punkt. Diese können auf verschiedenste Weise verwendet werden, um einen einmaligen Vorteil zu erhalten. Ein ausgegebener Karma-Punkt kann nicht zurückerhalten werden.
\subsubsection{Ein alter Gefallen}
Du kontaktierst eine Person aus deiner Vergangenheit, die dir noch einen Gefallen schuldig ist. Dieser Gefallen nimmt normalerweise die Form eines Stückchen Information, einer diplomatischen Entschärfung der Situation oder ein gewisses Maß an tatkräftiger Unterstützung an. Du könntest auch ein Stück Ausrüstung oder einen Gegenstand erhalten, solltest ihn aber ohne Kratzer zurückbringen!
\subsubsection{Plot-Armor} 
Wenn ein Charakter ausgeschaltet ist und durch eine Verletzung sterben würde, kann er einen Karma-Punkt dazu benutzen, um zu überleben und den Zustand \glqq Ausgeschaltet\grqq{} zu beenden. Er wählt zwei seiner Fähigkeiten und erhält eine zufällig wieder, die andere wird verkrüppelt.
\subsubsection{Flexibel} 
Tausche eine Fähigkeit deines Charakters aus, dieser Tausch dauert einen Zeitraum von 24 Stunden und wird bei schwerwiegenden Störungen, wie einem Kampf, abgebrochen, der Karma-Punkt geht aber nicht verloren.
\subsubsection{Wertewandel} 
Dein Charakter erhält in einem seiner Attribute \glqq +1\grqq{}, aber dafür in einem anderen Attribut \glqq -1\grqq{}.
\subsubsection{Wunderheilung}
Stelle am Ende einer Pause eine Fähigkeit wieder her, die bei dir Verkrüppelt wurde.
\section{Einen Charakter erstellen}
Diese Schritte sollten befolgt werden, um die Charaktere auf Level 0 zu erstellen.
\subsection{Charakter designen}
\begin{enumerate}
\item Wählen der Klasse des Charakters.
\item Festlegen der Charakter-Attribute. Unter Stärke, Geschicklichkeit, Konstitution, Intelligenz, Weisheit und Charisma werden \glqq +2\grqq{}, \glqq +1\grqq{}, \glqq +1\grqq{} und \glqq -1\grqq{} verteilt.
\item Wählen der Hintergrundfähigkeit.
\item Wählen der drei Fähigkeiten. Diese beinhalten sowohl Grund- als auch Spezialfähigkeiten.
\item Wählen eines Einsatz-Anzugs.
\item Notieren der Geschwindigkeit und des Sprints des Charakters.
\item Wählen von Anfangsgegenständen. (Der GM entscheidet über Gesamtwert und -anzahl)
\item Notieren der Boni auf Stärke(Nahkampf Abwehren)- und Geschicklichkeit(Fernkampf Ausweichen)-Checks und deren automatischen Wert (5 + Stärke/Geschicklichkeit + leere Inventar-Plätze + andere Boni).
\item Notieren von Angriffen und Aktionen, die ein Charakter regelmäßig ausführt.
\item Niederschreiben der Hintergrundgeschichte und des Aussehens, der Persönlichkeit, der Ideale und Wertvorstellungen und der Schwächen des Charakters.
\item Notieren von zwei Connections mit einem Loyalitäts-Wert von +1.
\end{enumerate}
\subsection{Charakter auswürfeln}
Ein Spieler kann seinen Charakter auch auswürfeln, wenn er einen ungewöhnlichen und interessanten Charakter erstellen will, oder ihm die Inspiration fehlt.
\begin{enumerate}
\item Würfeln der Klasse: Würfle 1d3, ordne dann das Ergebnis einer Klasse zu (Experte = 1; Kämpfer = 2, 3 = Überlebenskünstler).
\item Würfeln der Charakter-Attribute: Würfle 4d6, bis vier unterschiedliche Ergebnisse herauskommen. Ordne die Ergebnisse den 6 Charakter-Attributen zu (STÄ = 1; GES = 2; etc.) und teile dem niedrigsten \grqq +2\grqq{}, den beiden nächst höchsten \grqq +1\grqq{} und dem übrigen \grqq -1\grqq{} zu.
\item Würfeln der Hintergrundfähigkeit: Würfle 1d50, wähle dann die Fähigkeit mit derselben Nummer des Ergebnisses von der Liste der Charakter-Fähigkeiten.
\item Würfeln der drei Charakter-Fähigkeiten. Würfle 3d100, bis drei unterschiedliche Ergebnisse herauskommen. Wähle dann die Fähigkeiten mit derselben Nummer der Würfel-Ergebnisse von der Liste der Charakter-Fähigkeiten.
\end{enumerate}
Fahre nun mit dieser Basis bei Punkt 5 der Liste zur Charakter-Erstellung fort.

\section{Ein Level Aufsteigen}
Wenn Charaktere eine besonders schwere Mission oder einen Teil der Geschichte abgeschlossen haben, so steigen sie ein Level auf. Mit jedem neuen Level wird ein Charakter etwas mächtiger, bis zu einem Maximum von Level 5.
\begin{itemize}
\item Erhöhe den Wert eines Charakter-Attributs um 1 Punkt, bis zu einem Maximum von 5.
\item Erhalte einen Karma-Punkt.
\item Erreichen von Level 5: Wähle zusätzlich eine neue, vierte Fähigkeit.
\end{itemize}
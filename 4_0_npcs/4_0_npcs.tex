\chapter{Nichtspielercharaktere}
Die Charaktere der Spieler begegnen und interagieren ständig mit NPCs. Diese werden vom GM kontrolliert und weisen viele Eigenschaften auf, die Charaktere ebenfalls haben.
Der GM kann jederzeit einen Nichtspielercharakter auf dieselbe Weise erstellen und verwenden, wie die Spieler es mit ihren Charakteren tun. Dies empfiehlt sich vor allem für NPCs, die besonders starke Verbündete oder Gegner der Spielercharaktere darstellen oder anderweitig essentiell für die Kampagne sind.

Die Spieler begegnen aber Nichtspielercharakteren so oft, dass es zeitsparender und leichter für den GM ist, mit vereinfachten NPC-Bögen zu arbeiten. In diesen sind die Attribute, Fähigkeiten und Gegenstände eines Typus von Nichtspielercharakteren angegeben.
\subsubsection{Attribute des Nichtspielercharakters}
Die Attribute eines Nichtspielercharakters geben, genau wie bei Spielercharakteren, seine körperliche und geistige Leistungsfähigkeit an und werden für Checks verwendet. Ein Attribut, in das 0 Punkte investiert sind, entspricht der Leistungsfähigkeit einer gewöhnlichen Person. Viele Nichtspielercharaktere sind jedoch keine gewöhnlichen Personen und können eine Vielzahl von Boni auf ihre Attribute erhalten. Genauso können Charaktere auch einen negativen Wert in einem Attribut haben und so ein Malus auf einen Check erhalten.
\subsubsection{Geschwindigkeit}
Die Geschwindigkeit eines Nichtspielercharakters gibt an, wie weit er sich während seines Zuges bewegen kann. Dahinter in Klammern ist angegeben wie weit er kommt, wenn er seine Aktion zum Sprinten verwendet.
\subsubsection{Kampf}
Im gewöhnlichen Kampf machen die Nichtspielercharaktere aus NPC-Bögen keine Checks zum Angriff, sondern besitzen jeweils einen festgelegten Wert für die Checks beim Nah- oder Fernkampf (der Bonus auf den Check ist dennoch hinter diesem Wert in Klammern vermerkt). Dasselbe gilt für Nahkampf Abwehren- und Fernkampf Ausweichen-Checks, hier ist der automatische Wert des Checks (wenn der Nichtspielercharakter zum Beispiel ausgeschaltet ist) dahinter mit einem Schrägstrich getrennt angegeben.
Sollte ein Nichtspielercharakter Vorteil oder Nachteil bei einem Check erhalten, so würfelt er den Check normal mit 2d20. Für wichtige Nichtspielercharaktere empfiehlt es sich, sowieso alle Checks auszuwürfeln.
\subsubsection{Fähigkeiten}
Die Fähigkeiten eines NPCs dienen ähnlich wie die eines Spielercharakters gleichzeitig dazu, seine Trefferpunkte zu messen. Für jede Verletzung, die ein Nichtspielercharakter erhält, muss er eine Fähigkeit wegstreichen, und kann ihren Effekt nun nicht mehr nutzen. Hat er keine mehr übrig, so stirbt er gewöhnlicherweise. Manche wichtige NPCs können aber auch wie Spielercharaktere erst einmal ausgeschaltet werden.
Die Hintergrundfähigkeit(en) eines Nichtspielercharakters werden wie das eines Spielercharakters nicht weggestrichen, sondern stellt einen permanenten Bonus dar.
\subsubsection{Spezial}
Manche Nichtspielercharaktere haben besondere Fähigkeiten oder Eigenschaften, die unter diesem Punkt vermerkt sind.
\subsubsection{Ausrüstung}
Unter Ausrüstung wird die Kleidung/Ausrüstung vermerkt, die ein Nichtspielercharakter trägt, und der Effekt, den er von ihr erhält. Bietet die Ausrüstung keinen besonderen Bonus, so wird sie in Klammern angegeben.
\subsubsection{Inventar}
Im Inventar sind die Gegenstände vermerkt, die ein Nichtspielercharakter mit sich führt. Zuerst sind die Gegenstände angegeben, die er tragen kann, ohne dass sie seine (in Klammern angegebene) Tragelast beanspruchen. Danach kommen die anderen Gegenstände, bei denen vorne jeweils in Klammern ihr Gewicht angegeben ist.
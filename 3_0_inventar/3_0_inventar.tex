\chapter{Inventar}
\todo{Hinzufügen}
\section{Waffenarsenal}
\section{Gegenstände}
\section{Objekte}
\subsection{Eigenschaften von Objekten}
\subsubsection{Wert}
Manche Gegenstände kosten eine bestimmte Anzahl an Ressourcen (gewöhnlich eine gewisse Menge an Geld), die als abstrakte Zahl dargestellt ist und nicht der echten Summe entspricht. Ein Charakter kann Objekte aus der Liste der Gegenstände im Laufe eines Tages für gewöhnlich ohne Probleme und in ausreichenden Mengen erhalten, solange er Zugang zu einer gut entwickelten Infrastruktur hat. 
Manchmal ist der Wert eines Gegenstands nicht angegeben und somit vernachlässigbar, Charaktere können diese Gegenstände in angemessener Anzahl erhalten, ohne Ressourcen ausgeben zu müssen.
\subsubsection{Waffe}
Ein Charakter, der eine Waffe ausgerüstet hat, kann einen Stärke(Nahkampf Abwehren)-Check durchführen, um von einem Nahkampf-Angriff nicht verletzt zu werden. Waffen sind an den meisten zivilisierten Orten verboten.
\subsubsection{Reichweite}
Ziele außerhalb der Reichweite einer Fernkampfwaffe können nicht getroffen werden. Zudem hat ein Charakter bei seinem Check zum Angriff Nachteil, wenn sein Ziel 1 Länge oder weniger von ihm entfernt ist.
\subsubsection{Munition}
Die Zahl, die bei Munition angegeben ist, zeigt an, wie oft eine Waffe abgefeuert werden kann, bevor sie nachgeladen werden muss. Jedes Mal, wenn mit der Waffe ein Angriff ausgeführt wird, wird ihre Munition um mindestens 1 verringert.
\subsubsection{Verbrauchsgegenstand}
Ist ein Objekt ein Verbrauchsgegenstand, so wird es bei seiner Benutzung aufgebraucht und somit aus dem Inventar eines Charakters entfernt. Hat ein Verbrauchsgegenstand mehrere Trefferpunkte, so kann er unter den richtigen Umständen mehrmals benutzt werden.
\subsubsection{Drohne}
Eine Drohne ist ein unbemanntes kleines Fahrzeug, dass sich in einem Medium fortbewegen kann, dass beim Erhalten der Drohne festgelegt wird (Medien sind: Boden, Wasser, Weltraum, Luft und Unterwasser). Sie kann vom Handy eines Charakters oder einem vergleichbaren Gerät ferngesteuert werden, bewegt sie sich aber über die Maximalreichweite der Verbindung, so gilt die Drohne als ausgeschaltet. Genauso wird die Verbindung durch 1 Meter solides Gestein oder 10 Zentimeter Metall blockiert.

Jede Drohne hat eine Batterie, die nur für eine gewisse Zeit anhält. Ist sie aufgebraucht, so wird die Drohne ausgeschaltet. Besitzt eine Drohne einen Angriff, so nutzt sie für den Check die Geschicklichkeit des Charakters, der sie kontrolliert. Eine Drohne kann bei einem Check keinen Vorteil haben und versagt automatisch bei jedem Check, auf den sie Nachteil hat.
\subsubsection{Ausrüstung}
Ausrüstung kann von einem Charakter angelegt werden, und nimmt auf diese Wiese kein Gewicht weg. Ein Charakter kann immer nur eine Ausrüstung gleichzeitig angelegt haben und deren Bonus oder Effekt erhalten. Das An- und Ablegen einer Ausrüstung benötigt mindestens 1 Aktion.
Das von Spielercharakteren am häufigsten getragene Ausrüstungsstück ist der Einsatz-Anzug.
\subsection{Herkunft von Gegenständen}
Gebrauchsgegenstände kommen nicht aus dem nichts, sondern werden normalerweise von Firmen hergestellt und verkauft. Praktisch jeder Charakter würfelt seine Ausrüstung aus den jeweiligen Topsellern der Marktführer zusammen, und diese sollte man alle sowieso auswendig aufzählen können, denn diese Namen stehen inzwischen für mehr als nur ihre Produkte: Wird eine dieser großen Firmen erwähnt, dann wird die Sache erst richtig interessant!

Ein Spieler muss nicht notieren, welche Marke die Socken sind, die sein Charakter trägt. In einer globalen Wirtschaft interessiert es meistens keinen, wo etwas herkommt, weil man eh überall Alles bekommen kann. Aber bei bestimmten Schlüsselgegenständen kann es einen Spielercharakter um einiges interessanter machen (meist in Zusammenarbeit mit der Hintergrundgeschichte). Verabscheut ein Charakter Aztechnlogy, oder bleibt er dem Konzern zu 100 Prozent markentreu? Ist es die beste Idee, Schutzwesten von Friedl-Fuchi anzuziehen, wenn man bei Zinke Megastrukturen mal eben ein paar Safes ausräumen will? Oder lässt man absichtlich einen THI-Kugelschreiber fallen, um eine falsche Spur zu legen? Solche Überlegungen sind kein Muss, aber auf jeden Fall die Mühe wert!
\subsubsection{Der Instanz-Rat}
Gegen Mitte des 21. Jahrhunderts brach die schwächelnde Weltordnung der Nationalstaaten endgültig zusammen. Befreit von Landesgrenzen erhoben sich die Megafirmen wie gierige Phönixe aus der Asche der Weltwirtschaft und übernahmen persönlich die Kontrolle, die sie sich früher mühsam in der Lobby hatten erarbeiten müssen.

Ohne die althergebrachten Bündnisse, Verhaltenscodexe und Vorurteile brach in der Welt Chaos aus, als verschiedenste Fraktionen durch wackelige Bündnisse möglichst viel Macht ansammeln wollten. Nach einigen Jahren hatte sich ein Kern aus starken und unabhängigen Firmen und Koalitionen gebildet, die ihre neugewonnene Position auf keinen Fall aufgeben wollten und so eine Institution gründeten, die die Ordnung und Sicherheit auf der ganzen Welt gewährleisten sollte (zu Gunsten ihrer Gründer, natürlich). Somit war der Instanz-Rat geboren.

Jede größere Organisation kann eine Anzahl an den Sitzen im Rat beantragen, die dann auf Basis der Vermögensverhältnisse und des Einflusses vergeben werden. Die meisten Sitze hat momentan Friedl-Fuchi inne, gefolgt von Aztechnology mit seinen zahllosen Unterfirmen. Während die oberen Positionen im Rat relativ gefestigt sind, kommt es am unteren Ende der Sitzverteilung zu einem ständigen, gnadenlosen Kampf. Alles wird getan, um die Hürde für mindestens einen Sitz zu überschreiten oder die Konkurrenz so stark zu schwächen, dass sie ihre Position verliert. Am Ende einer der 5-jährigen Wahlperioden kommt es zu einer regelrechten Explosion von Werbekampagnen, Produkterscheinungen, Free Trials und Innovationen, wenn alle Firmen versuchen, genug Umsatz zu machen, um in den Instanz-Rat zu gelangen. Manche Personen, genannt \glqq Surfer\grqq{}, reiten buchstäblich auf diesen Wellen um ein Vermögen zu verdienen (anstelle es rechtschaffend zu erben oder erarbeiten - deshalb sind Surfer meist recht unbeliebt).

Wer im Rat die Mehrheit hat, prägt auch die Weltpolitik. Entsprechend den Wünschen und Bedürfnissen der großen Firmen werden Subventionen und Regulierungen eingesetzt (aber keiner darf sich zu viele Monopole erarbeiten, da er sofort den ganzen Rest des Rates gegen sich hätte). Aus Publicity-Gründen wird dabei sogar ein Abglanz von Moral und Grundrechten sichtbar. Keine Macht kann zu lange gegen die Rechtsprechung des Rates ankämpfen, da die gesammelte Konkurrenz nur auf einen legitimen Grund wartet, um eine Feindfirma vom Markt zu bomben (der Feind meines Feindes ist schließlich mein Freund). Und so hält sich die Situation mehr oder weniger in der Waage, da alle großen Firmen sich in ihre Nischen und Territorien eingelebt haben.

Die Mitglieder des Rates treffen sich niemals alle in Person, sämtliche Sitzungen werden über das Internet (natürlich mit den besten Sicherheitsvorkehrungen, die sich die reichsten Leute der Welt leisten können) gehalten. Dies soll gleichzeitig das Anschlagsrisiko minimieren und als Symbol dafür stehen, dass der Rat keinen bestimmten Ort auf der Welt bevorzugt. Die Gesetze des Instanz-Rat werden in allen Sprawls der Erde von den jeweiligen Sicherheitskräften durchgesetzt, mit teilweise zweifelhaftem Erfolg. In den äußeren Zonen des Sonnensystems verliert der Rat aber an Macht, weshalb dort meistens \glqq lokale\grqq{} Gesetzgebung angewendet wird.
\subsubsection{Friedl-Fuchi}
Umgangssprachlich auch \glqq FF\grqq{}, \glqq Doppel F\grqq{} oder \glqq Fuckers\grqq{}
Es ist ein einzigartiges Privileg, an der Spitze der Welt zu stehen, und Friedl-Fuchi nutzt diesen Vorteil aufs Vollste aus. Die größte und einflussreichste Megafirma im ganzen Sonnensystem kennt kein Pardon, denn sie wurde dank Rücksichtslosigkeit, Aggressivität und Gnadenlosigkeit zum größten Fisch im Teich. Ihre bescheidenen Anfänge hatte Friedl-Fuchi in Europa, wohl um 2020 herum, als ein Zusammenschluss von Unternehmen in der Robotik, der Metallindustrie und dem Autobau. Durch das unglaubliche Geschick des ersten Geschäftsführers arbeitete sich diese Ur-Firma zu einer internationalen Größe herauf. Die genauen Details dieses Aufstiegs sind heutzutage unbekannt, da fast alle Hinweise auf die Ursprünge von Friedl-Fuchi ausradiert wurden, voller Widersprüche stecken oder zugangsbeschränkt sind.

Um 2035 hatte Friedl-Fuchi eine dominante Position im europäischen und asiatischen Raum eingenommen, und expandierte unter der neuen Direktorin Vanessa Zhang mit schier unersättlichem Ressourcenhunger nach Afrika. Die Megafirma schien unaufhaltsam, doch dann wurde 2042 die Europazone neugegründet. Die europäischen Instanzen wollten ihre Macht nicht an einen Emporkömmling verlieren, und nach einem harten Sanktions- und Reformenkrieg (mit allzu eifriger Hilfe von Aztechnology) wurde Friedl-Fuchis Machtbasis ausgehebelt. Als Resultat dieser Politik wurden praktisch alle Friedl-Fuchi-Niederlassungen in Europa aufgekauft. Die Megafirma passte sich jedoch innerhalb weniger Jahre an und verlegte ihren Hauptsitz nach Japan. Inzwischen gibt es in Südasien und auf dem ganzen Kontinent Afrika keine Regierung mehr, die nicht von Friedl-Fuchi kontrolliert wird. Die Gebiete wurden in die sogenannten Dodecanate aufgeteilt, von denen jedes ein (vom Firmenvorstand aufgestelltes) demokratisch gewähltes Team von 3 Parteiführern besitzt. Diese übernehmen die lokale Politik und haben ihre Sitze jeweils in drei Großstätten ihres Bezirks. Letztendlich müssen sie sich aber vor Friedl-Fuchi verantworten, und obwohl die Parteiführer zu extremen Rivalitäten neigen, haben sie keine Wahl, als an einem Strang zu ziehen, wenn ihre Chefs es von ihnen fordern.

Friedl-Fuchi ist weltweit Marktführer in der Schwerindustrie, hat sich aber eine unglaublich breite Basis aufgebaut. Bei jeder Unternehmung hat Friedl-Fuchi irgendwie seine Finger im Spiel. Nicht selten werden die Strukturen von geschluckten Firmen oberflächlich beibehalten, aber alle Chefpositionen neu besetzt. Trotz der rücksichtslosen Firmenpolitik steht es bei Friedl-Fuchi an erster Stelle, Sicherheitsbestimmungen, Arbeitsverträge und Renten ihrer Mitarbeiter immer genauestens einzuhalten. Doch nach außen zeigt die Megafirma von dieser Fairness kaum etwas, am wenigsten ihren Konkurrenten.
\subsubsection{Bodo Presssack-Foundation}
Umgangssprachlich auch \glqq Pressäcke\grqq{}, \glqq Bodonier\grqq{}
Der Gründer dieser raumgebundenen Megafirma, Bodo Presssack, galt schon immer als ein exzentrischer Eigenbrötler, auch schon auf der Erde. Sein Unternehmen war auf die Entwicklung von revolutionären Düngetechniken und Genmanipulation von Pflanzen spezialisiert, und erreichte so während der Versorgungskrisen schnell einen hohen Marktwert. Kurz darauf kam es vermehrt zu Versuchen der feindlichen Übernahme durch mächtige Konkurrenten, dann investierte Bodo plötzlich einen Großteil seiner Ressourcen in die Raumfahrttechnik. Sein Team begann damit, sein ganzes Imperium vom Planeten Erde in den Raum hinauf zu verschiffen. Inzwischen ist seine Firma komplett im interplanetaren Raum angesiedelt, und hat zwei Haupteinnahmequellen: Die Erzeugung von Lebensmitteln und Biomasse auf Raumstationen und den (auf Wunsch unauffälligen) Transport von Personen durch das Sonnensystem. Operationsbasis dieses Geschäfts ist der \glqq Presssack\grqq{}, eine gigantische Raumstation im Lagrange-Punkt L3 des Sonne-Erde Systems. Von hier aus übt seit Bodos natürlichem Tod die KI Merlin, die er eigenhändig aufgezogen haben soll, die Kontrolle über sein Imperium aus.

In der Bodo Presssack-Foundation zählen die Rechte des Individuums zur freien Entfaltung, ohne irgendwelche Vorurteile, eine große Rolle. Es wird von Abteilungsleitern der Firma gefordert, bei ihren Mitarbeitern ein Gefühl von Familie und Gemeinschaft (und eine gewisse Schrulligkeit) zu wecken. Die Hierarchie der Foundation ist sehr flach, und alle Angestellten nehmen regelmäßig an Mitarbeiterbefragungen auf ihrem \glqq Bodophon\grqq{} teil, bei denen sie aktiv den Kurs der Megafirma beeinflussen können. Dennoch ist es ratsam, Vorsicht walten zu lassen, denn die Fima mag zwar sehr aufnahmefreundlich sein, lässt aber Verbündete dann nur ungern wieder gehen. Zudem hat die Foundation ein langes Gedächtnis (in Form von umfassenden Datenbanken über alle Mitarbeiter und jede externe Zusammenarbeit), und jegliche Aktion gegen sie wird mit Sicherheit irgendwann vergolten werden.
\subsubsection{Toha Heavy Industries}
Umgangssprachlich auch \glqq THI\grqq{}, \glqq Cyborg City\grqq{}
Eine der Megafirmen mit der ältesten Konzerngeschichte ist Toha Heavy Industries, welche in der damaligen Volksrepublik China schon Mitte des zwanzigsten Jahrhunderts gegründet wurde, aber lange Zeit unter dem Radar blieb. Ursprünglich war sie auf Schiffs- und Ölbohrplattformen-konstruktion spezialisiert. Nach dem Wirtschaftskrach von 2022 nutzte die Firmenleitung eine Chance, mit einem Bio- und Computertechnologie-Konglomerat zu verschmelzen und den Namen (kurz THI) beizubehalten. Von hier an konnte sich die Firma durch extrem hohe Produktqualität und einem gewissen Talent für die Fertigung von Einzelstücken (es wurde stets versucht, nur die talentiertesten und erfahrensten Mitarbeiter anzuwerben) ihren Platz auf dem Weltmarkt sichern. Schließlich sollte sie aber von ihrer dominanten Position im Osten von Friedl-Fuchi gestürzt werden, als die Megafirma ihre Präsens aus Europa zurückziehen musste. Toha Heavy Industries gelang es jedoch, mit FF zu koexistieren und die beiden Firmen arbeiten inzwischen in der Regel zusammen. Jetzt ist der Konzern ein Hochpräzisionsinstrument, das vor allem Sonderanfertigungen vertreibt.

In der Entwicklung und Produktion von Cyborgs, Bio-Interfaces und Hochleistungsrechensystemen macht ihnen keine andere Megafirma etwas vor. Auch wurde sie von Friedl-Fuchi mit der Planung mehrerer Kolonien außerhalb der Erde beauftragt, die den höchsten Lebensstandards entsprechen sollten. Besonders die Kolonie auf dem roten Planeten macht THI zu einem der größten Konzerne außerhalb der Erde.

Bei Toha Heavy Industries zeigt man (vor allem in der Chefetage) noch immer einen gewissen Stolz in Bezug auf die fernöstliche Herkunft der Firma. Asiatische Philosophien werden \glqq modernisiert\grqq{} und direkt auf die Firmenpolitik angewandt und der Konzern gilt als einer der wenigen \glqq fairen\grqq{} Spieler auf dem Weltmarkt. Dem Firmenrat werden jedoch immer wieder Anschuldigungen der Starrköpfigkeit und gar des Rassismus (besonders gegenüber nichtmenschlichen Personen) vorgeworfen. Und zudem würde sich THI ohne die Kooperation mit Fried-Fuchi nicht lange auf dem Markt halten können, denn ihre Produkte sind einfach zu spezialisiert und kostspielig.
\subsubsection{Aztechnology}
Umgangssprachlich auch \glqq Amerika\grqq{}, \glqq Kraken\grqq{}, \glqq Bigheads\grqq{}
Die vereinigten Staaten von Amerika hatten große Probleme, sich an die neue, direkt (und nicht mehr nur durch Lobbyismus) von Firmen kontrollierte Ökonomie anzupassen. Das Bruttoinlandsprodukt sank auf Rekordtiefen, Sozial- und Gesundheitsversorgung wurden unbezahlbar. Viele Millionäre und Firmen hatten bereits damit begonnen, ihre Standorte und Wohnsitze anderswo hin zu verlegen. An diesem Punkt boten Vertreter des Softwareunternehmens Aztechnology der kränkelnden Regierung ihre Fähigkeiten an. Im Austausch Monopolrechte und Subventionen würde die Firma (mit hochleistungsfähigen Wirtschaftsanalyseprogrammen, verbesserten Transport- und öffentlichen Verkehrsmitteln und einem effizienten Sicherheitssystem) die Wirtschaft der einst mächtigsten Nation wieder ankurbeln. Obwohl dieses Angebot zuerst auf viel Wiederstand in der Regierung traf und eine Vielzahl von Konkurrenzfirmen die Kollaboration zu verhindern versuchten, wurde schließlich der Vertrag geschlossen. Und damit war der Aufstieg von Aztechnology in der westlichen Welt unaufhaltsam.

Aztechnology ist das Nervensystem eines gigantischen Firmen-Organismus, denn der Konzern kontrolliert den Informationsaustausch und die Aufgabenverteilung auf dem ganzen amerikanischen Kontinent. Der Fokus liegt hierbei vor allem auf Konsum- und Alltagsgüter. Basierend auf den Entscheidungen von endlosen Algorithmen werden Unter-Firmen gefördert oder vernachlässigt, Personen aufgenommen oder ausgewiesen, Wirtschaftszweige ausgeweitet oder abgebaut. Aztechnology spielt hierbei nur der Mittelmann, aber in diesem Fall übt der Mittelmann Kontrolle über einen ganzen Kontinent aus. Aus diesem Grund besitzt Aztechnology auch von allen wichtigen Megafirmen bei weitem die kleinste Mitarbeiteranzahl, da die \glqq eigentliche\grqq{} Arbeit auf eine Vielzahl von kleineren Firmen verteilt wird. Die Bürger wählen zwar noch einen amerikanischen Präsidenten, dieser ist aber Aztechnology verpflichtet (beziehungsweise den Algorithmen, von denen gemunkelt wird, dass nicht mal ihre Erschaffer sie noch ganz im Griff haben). Der Einfluss der Megafirma reicht nicht viel weiter als ihr Datennetzwerk, weshalb sie auch nur geringen Einfluss außerhalb der Erde hat.
\subsubsection{Zinke Megastrukturen}
Umgangssprachlich auch \glqq Grey\grqq{}, \glqq Troopers\grqq{}, \glqq Giants\grqq{}
Zinke Megastrukturen ist die einzige der Megafirmen, die gar nicht erst versucht, ihre brutale und rücksichtslose Vorgehensweise zu vertuschen. Ursprünglich war der Konzern Zinke Safety Services ein Ausbilder und Vermittler von Sicherheitskräften und elite-Bodyguards, die ihrem Ruf von kalter Effizienz und Skrupellosigkeit mehr als gerecht wurden. Ein guter Teil der Streitkräfte der ersten großen Firmenkonflikte stammten direkt oder indirekt von Zinke Megastrukturen, und die ständig steigende Nachfrage ließ den Wert des Unternehmens in die Höhe schnellen. Inzwischen ist der Konzern mehr eine private Militärmacht als eine Sicherheitsfirma, was den Generalstab auch dazu anregte, ihre Ressourcen auf einen zweiten Produktionszweig aufzuspalten: Gigantische Kriegsmaschinerien, die die \glqq Balance of Power\grqq{} aufrechterhalten sollten. Und so wurde der Konzern in Zinke Megastrukturen umbenannt, die Produktionskette umfasst nun Spezial-Panzer, Kriegsschiffe, Raumfahrzeuge, Festungen, Bunker und die berühmten Cyber-Zeppeline (das neue Markenzeichen der Firma). Mit der Zeit wurde der Konzern so etwas wie eine Anlaufstelle für Waffenentwickler und Wissenschaftler, die ihre exotischeren Visionen in anderen Firmen nicht umsetzen durften.

\glqq Präzise und Gründlich, darauf ist Verlass!\grqq{} (Solang man es sich leisten kann!) ist der Leitsatz des Konzerns, der sein Image gegen die Unberechenbarkeit und Korruption der anderen Firmen absetzt. Jede Sicherheitskraft führt jede Anweisung gewissenhaft und ohne lästige Fragen durch. Dieser Ruf hat in gewissen Kreisen zu einem recht hohen Ansehen dieser Firma geführt, deren einziges Produkt Kriegsmaschinerien sind, und für die Moral ein Fremdwort darstellt. Tatsächlich liegt die Treue jeder Sicherheitskraft und jedes Bodyguards (dank effizienter \glqq Trainingsmethoden\grqq{}) allein bei Zinke Megastrukturen. Genauso hat jedes Kriegsfahrzeug und jede Struktur eine Hintertür oder Schwachstelle, die nur dem Herstellerkonzern selbst bekannt ist. So kann sich die Firma im Zweifelsfall immer einen diskreten Vorteil verschaffen, doch geht sie hiermit vorsichtig um, damit ihrem Ruf nicht geschadet wird.
\subsubsection{AmaZone}
Umgangssprachlich auch \glqq Oldies\grqq{}, \glqq Geldsäcke\grqq{}
Das Relikt einer vergangenen Zeit unter den Megafirmen ist AmaZone, das große Joint Venture. Mit den raketenhaft aufsteigenden Neufirmen konnten Unternehmen des 20. und frühen 21. Jahrhunderts nur schwer mithalten. Deshalb schlossen sich viele der größten Banken und Gesellschaften und auch \glqq Firmen-Stars\grqq{} wie Amazon, Apple, und Monsato (um einige wenige zu nennen) zu einem großen Konglomerat zusammen. Nach endlosen internen Streitigkeiten, 34 Namensänderungen, mehreren Kleinkriegen und Jahren der Unproduktivität wurden die einzelnen Fraktionen schließlich unter der Familie Uriarte geeint oder schlicht aufgekauft, und so endlich Ordnung in das Chaos gebracht. Nach einigen Neustrukturierungen und kostspieligen Marketingkampagnen war die Leistungsfähigkeit und das Image der Megafirma wieder so weit aufgebaut, dass sie sich wieder der Konkurrenz zuwenden konnte.

Der Konzern ist global vertreten, Hauptniederlassungen befinden sich in Neu-Delhi, London, Seattle, Rom, Sydney und etlichen weiteren Sprawls. Genauso weit gestreut ist das Angebot der Firma, sie \glqq umfließt\grqq{} ihre Konkurrenz und füllt jede sich darbietende Marktlücke schnellstmöglich aus.
AmaZone gilt als einer der \glqq schwächsten Charaktere\grqq{} unter den Megafirmen, was jedoch vor allem daran liegt, dass ein guter Teil der Umsätze nicht klar ersichtlich und der ganze Firmenkomplex total unüberschaubar ist (es kommt durchaus des Öfteren vor, dass sich Unterfirmen von AmaZone gegenseitig ausstechen und schaden). Der Konzern besitzt wenig konzentrierte Macht, kann aber auf breiter Ebene und besonders unauffällig agieren. Die sich im Schatte ihrer Mega-Konkurrenz haltende Firma wartet lieber geduldig ab, bis ihre Zeit gekommen ist, um ein Comeback zu machen.
\subsubsection{Australo-Pazifisches Bündnis}
Australien blieb einer der am wenigsten besiedelten Kontinente, auf dem Sprawls, wenn überhaupt, nur an den Küsten wirklich vorkommen. Dies wird als einer der Hauptgründe gesehen, weshalb die Megafirmen hier nie so Fuß fassen konnten wie auf der restlichen Welt. Um aber politisch relevant bleiben zu können, schlossen sich unter anderem Australien, Neuseeland, Indonesien, Thailand und Vietnam zum Australo-Pazifischen Bündnis zusammen. In diesem Gebiet entspricht die politische Landschaft am ehesten noch der der Nationalstaaten vor der Machtübernahme der Megafirmen. Doch natürlich zeigen die Konzerne auch hier ihre Präsens (und konkurrieren um den Absatzmarkt), vor allem Friedl-Fuchi ist an den Gebieten der Bündnisses sehr interessiert.

Alle drei Jahre wird im Australo-Pazifischen Bündnis ein neuer Präsident gewählt, der momentane politische Trend entspricht gerade \glqq leben und leben lassen\grqq{}, ist aber auch sehr isolationistisch. Die Immigrationsbedingungen sind hart und die Landesgrenzen werden eifersüchtig bewacht.
\subsubsection{Europazone}
In der Europäischen Union entstand der Zündfunke, der fast zu einem Dritten Weltkrieg und nun zur Herrschaft der Megafirmen geführt hat. In den Zeiten von wirtschaftlicher Instabilität, Naturkatastrophen dank Klimawandel und aggressivem Populismus konnten und wollten die alten Nationalstaaten nicht mehr zusammenarbeiten und sich regulieren lassen, und Europa zersplitterte in einen Flickenteppich aus hastigen Allianzen und Anfeindungen, doch der große Endschlag blieb aus. Die einzelnen Staaten waren zwar jeweils sehr von sich überzeugt und enthielten oft viele extremistische Gruppen, doch voneinander isoliert war ihre Macht auf dem Weltmarkt vernachlässigbar. Und so, bevor ein totaler Krieg ausbrechen konnte, übernahmen verschiedene Megafirmen die Regierungen (meist, ohne großes Aufsehen darum zu machen). Die Spannungen zwischen den Staaten erstarben (auch wenn Terrorismus ein beständiges Problem darstellt), und wurden durch das verdeckte Intrigenspiel der Konzerne ersetzt. Praktisch jeder große Konzern übt hier direkt oder indirekt seine Kontrolle aus und versucht, seine Konkurrenz auszumanövrieren.

Europa gilt immer noch als ein Zentrum der Kultur, auch wenn historischen Stätten vor allem in Küstennähe dank steigendem Meeresspiegel und durch Verschmutzung stark zugesetzt wurde. Das Inland des Kontinents ist von einigen der größten Sprawls der Welt durchzogen, bewohnt von den landesflüchtigen Europäern und den Nachkommen der Völkerwanderung aus dem Mittleren Osten und Nordafrika nach Europa zu Beginn des 21. Jahrhunderts. Der Kontinent ist ein chaotischer Schmelztiegel der Kulturen mit krassen Gegensätzen zwischen Arm und Reich, der keine Zugehörigkeit zu einer der Machtstrukturen hat, auch wenn seine Bewohner es sich gerne einbilden. Die alten Nationalsaaten treten hier als Geister ihrer selbst auf, verkalkte Marionetten der Konzerne ohne eigene Kontrolle, erhalten wie in einem Museum der früheren Fehler der Menschheit (mit dem die Firmen aber auch zeigen, wie wenig sie die Situation verbessert haben).
\section{Fahrzeuge}
Jedes Fahrzeug hat eine festgelegte Anzahl von Trefferpunkten. Hat es keine mehr übrig, so wird das Fahrzeug zerstört. Ein Fahrzeug erzeugt am Beginn seines Zuges eine feste Anzahl an \glqq Energie\grqq{}, die es auf verschiedene Systeme aufwenden kann. Ein Charakter wird als der \glqq Fahrzeugführer\grqq{}  oder Kapitän bezeichnet. Das Fahrzeug handelt an seinem Zug, und Systeme, die von anderen Charakteren kontrolliert werden, handeln ebenfalls während dieses Zuges (erhalten aber dann in dieser Runde keinen eigenen Zug).

Charaktere, die sich in einem Fahrzeug befinden und nicht ihre Aktion dazu benutzen, ein System zu kontrollieren, können an ihrem Zug normal agieren. Solange sich ein Charakter in einem Fahrzeug befindet, hat er automatisch halbe Deckung.

Ein Fahrzeug kann keinen Nahkampf Abwehren- oder Fernkampf Ausweichen-Check durchführen. Sein automatischer Nahkampf Abwehren- und Fernkampf Ausweichen-Wert ist 14.
Die meisten Waffen fügen einem normalen Fahrzeug 1 Punkt schaden zu, außer wenn bei der Waffe anders beschrieben.
\subsubsection{Systeme}
Ein Fahrzeug hat eine feste Anzahl an Systemen, von denen jedes eine bestimmte Anzahl an Energie kostet und beliebig oft in einer Runde eingesetzt werden kann, bis die Energie des Fahrzeugs erschöpft ist. Normalerweise dienen diese Systeme entweder der Fortbewegung oder dem Angriff.

Für jedes einzelne System, das von einem Fahrzeug betrieben werden soll und das dazu Energie verbraucht, muss sich mindestens ein Charakter an Bord befinden, der seine Aktion dazu verwendet, dieses System zu kontrollieren. Ein Fahrzeug kann sich also in einem Zug fortbewegen, solange sich ein Charakter am Steuer befindet und seine Aktion dazu benutzt. Um aber eine Waffe im selben Zug abzufeuern, muss ein zweiter Charakter in dem Fahrzeug während dieses Zuges zudem seine Aktion auf die installierte Bewaffnung verwenden.
Beim Erhalten eines Fahrzeugs wird festgelegt, welche Systeme es besitzen kann. Zudem ist die Anzahl der Systeme pro Fahrzeug limitiert.
\subsubsection{Größe}
Die Größe eines Fahrzeugs ist der Bereich auf dem Spielfeld, den es während eines Zuges näherungsweise einnimmt.
\subsubsection{Stauraum}
Der Stauraum eines Fahrzeugs entspricht dem gesamten Platz, den es Passagieren und Fracht bieten kann. Jeder Punkt an Stauraum entspricht einem Gegenstand mit dem Gewicht von 1.
Ein Charakter nimmt immer 5 Stauraum weg, egal wie viele und welche Gegenstände sich in seinem Inventar befinden.

Ein Gegenstand, der sich auf dem Fahrzeug befindet und über den Stauraum hinaus Gewicht hinzufügt, fällt am Ende des nächsten Zuges des Fahrzeugs von diesem herunter. Ebenso muss jeder überschüssige Charakter am Ende des Zuges des Fahrzeugs einen Schwierigkeitsklasse 15 Geschicklichkeit-Check durchführen, oder ebenfalls abgeworfen werden.
\subsubsection{Gepanzert}
Ein gepanzertes Fahrzeug kann von außen nicht von Nahkampfwaffen, improvisierten Waffen, flächenwirkenden Waffen und leichten oder beidhändigen Fernkampfwaffen beschädigt werden. Charaktere innerhalb des Fahrzeugs haben volle Deckung gegenüber Angriffen von außen.
Faustregel: Wenn bei einem Angriff ein Check mit einem d20 gemacht werden muss, um überhaupt Schaden zu verursachen, dann hat dieser Angriff auf ein gepanzertes Fahrzeug keine Wirkung.
\subsubsection{Fortbewegung}
Ein System eines Fahrzeugs erlaubt es ihm immer nur, sich in oder auf einem Medium fortzubewegen. Die Medien sind Boden, Wasser, Weltraum, Luft und Unterwasser (siehe \secref{sec:fliegenundtauchen}).
Ein Fahrzeug im falschen Medium hat eine Geschwindigkeit von 0 Längen.
\subsubsection{Rammen}
Ein Fahrzeug kann während seines Zuges ein System, das es fortbewegt, dazu benutzen, ein anderes Fahrzeug zu rammen. Hierfür kann es einmal pro Runde folgenden Angriff ausführen:
Rammen: Energiekosten: -; Reichweite: Nahkampf;
Schaden: (Anzahl der zurückgelegten Längen / 2) – 1d4, Schaden wird gleichmäßig zwischen Angreifer und Ziel aufgeteilt. Ungerader Restschaden kann vom Angreifer verteilt werden.
\subsubsection{Zerstören von Fahrzeugen}
Werden die Trefferpunkte eines Fahrzeugs auf null reduziert, so wird es zerstört und an derselben Stelle bleibt nur noch ein Wrack zurück. Jeder Charakter, die sich zu diesem Zeitpunkt in ihm befindet, macht einen Schwierigkeitsklasse 10 Check mit einem Bonus von +2 für jedes Charakterlevel, dass er hat. Gelingt der Check, so landet der Charakter nach dem Angriff, der das Fahrzeug zerstört hat, innerhalb von 1 Länge unbeschadet neben dem Fahrzeug. Schlägt der Check fehl, so erleidet er eine Verletzung und landet innerhalb von 1 Länge neben dem Fahrzeug.
\section{Charaktere und Fahrzeuge}
\subsubsection{Ungeschützte Charaktere}
Wird ein Charakter ohne Einsatz-Anzug oder vergleichbaren Schutz mit der Bewaffnung eines Fahrzeugs angegriffen, so kann er einen tödlichen Schwierigkeitsklasse 12 Geschicklichkeit(Fernkampf Ausweichen)-Check versuchen. Bei einem Erfolg kann der Charakter seiner Vernichtung entkommen, bei einem Fehlschlag bleibt meist nicht viel von ihm übrig.
\subsubsection{Charaktere in Einsatz-Anzügen}
Trägt ein Charakter einen Einsatz-Anzug oder vergleichbaren Schutz und wird mit der Bewaffnung eines Fahrzeugs angegriffen, so kann er einen gefährlichen Schwierigkeitsklasse 12 Geschicklichkeit(Fernkampf Ausweichen)-Check versuchen. Bei einem Erfolg weicht der Charakter dem Angriff aus, bei einem Fehlschlag wird er verletzt. Kann der Angriff mehr als einen Schaden verursachen, so erleidet der Charakter für jeden Punkt an verursachten Schaden eine Verletzung, er erleidet aber immer mindestens eine.
\subsubsection{Charaktere in der Nähe von Fahrzeugen}
Während eines Kampfes bewegen sich Fahrzeuge oft mit hohen Geschwindigkeiten. Personen, die absichtlich oder unabsichtlich Gefährten zu nahekommen, können leicht überfahren oder erfasst werden. Genauso können die Piloten von Fahrzeugen versuchen, ihre Gegner auf diese Weise anzugreifen.

Ein Charakter, der selbst nicht in einem Fahrzeug ist und bis auf eine Länge an ein Fahrzeug, dass sich in seinem letzten Zug bewegt hat, herankommt, muss einen Schwierigkeitsklasse 10 Geschicklichkeit(Akrobatik)-Check durchführen, um auszuweichen. Schlägt dieser fehl, so wird der Charakter verletzt und kann während dieses Zuges dem Fahrzeug aus eigener Kraft nicht näher als eine Länge kommen.

Wenn sich ein Fahrzeug während seines Zuges bis auf eine Länge an einen Charakter heranbewegt, der sich nicht selbst in einem Fahrzeug befindet, kann es diesen auf dieselbe Weise verletzen. Während seines Zuges kann ein Fahrzeug einen Charakter so nur einmal zu einem solchen gefährlichen Geschicklichkeit(Akrobatik)-Check zwingen. Es können aber mehrere Fahrzeuge im Laufe einer Runde einen Charakter oder ein Fahrzeug mehrere verschiedene Charaktere in einem Zug verletzen.
\subsubsection{Gepanzerte Fahrzeuge beschädigen}
Viele größere Fahrzeuge gelten als Gepanzert, und werden somit von den meisten Waffen, die von Charakteren getragen werden können, nicht beschädigt. Manche Objekte wie Schwere Fernkampfwaffen, Handgranaten, Tretminen und Sprengstoff können gemäß ihrer Beschreibung aber Fahrzeuge angreifen. Mit diesen Waffen muss ein Charakter dann keinen Check mit einem d20 durchführen, um das Ziel zu treffen. Es muss sich nur innerhalb der Reichweite befinden und es darf keine volle Deckung vorhanden sein. Wichtig zu beachten ist, dass die Waffen hier NICHT mindestens 1 Schaden verursachen, wenn das Fahrzeug gepanzert ist.
Nicht Gepanzerte Fahrzeuge können auch einfach mit einem Check und jeder herkömmlichen Waffe angegriffen werden (Automatischer Wert des Fahrzeugs ist 14).

\todo{Grundkategorien}
\todo{Beispielfahrzeuge}

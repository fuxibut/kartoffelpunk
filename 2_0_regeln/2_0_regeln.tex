\chapter{Spielregeln}
Mehrere Spieler treffen sich zu einer Session. Ein Spieler, der Gamemaster (kurz GM), leitet das Spiel an. Er beschreibt die Welt, kontrolliert die Nichtspielercharaktere (NPCs) und agiert als Schiedsrichter zwischen Spielern und Regelwerk. Die Spieler übernehmen die Rolle ihres Charakters, und beschreiben dem GM die Aktionen ihrer Charaktere, der ihnen wiederum die Folgen dieser Aktionen erklärt, und so weiter.
\textbf{Jeder am Tisch ist dafür verantwortlich, dass sich alle wohl fühlen und Spaß haben!}
\section{Checks}
Ein Check muss immer dann von einem Charakter durchgeführt werden, wenn er eine Handlung in einer schwierigen oder gefährlichen Situation ausführen will oder muss. Normalerweise entscheidet der GM, wann und was für einen Check ein Spieler durchführen muss. Um einen Check durchzuführen, würfelt ein Spieler 1d20 und addiert darauf alle Boni, die er auf den Check durch seinen Charakter, Ausrüstung, Umgebung oder andere Umstände erhält. Dieser Wert wird dann mit der Schwierigkeitsklasse des Checks verglichen. Der GM gibt vor einem Check an, um welche Art von Check es sich handelt und legt die Schwierigkeitsklasse fest. Beträgt der Wert des Checks mindestens die Höhe der Schwierigkeitsklasse, so ist er ein Erfolg.

Für Checks, die Charaktere betreffen, wird normalerweise ein Charakter-Attributs-Check durchgeführt (z.B. ein Stärke-Check). Hat ein Charakter eine Fähigkeit, von der er meint dass sie auf den Check anwendbar ist, so kann er dies dem GM mitteilen, welcher dem Spieler eventuell einen Bonus oder Vorteil auf den Check gewährt. Wird für einen Check eine spezielle Grundfähigkeit genutzt, so wird dieses hinter dem Charakter-Attribut in Klammern angegeben (z.B. ein Stärke(Nahkampf Abwehren)-Check, um von einem Nahkampfangriff nicht verletzt zu werden). Ein Charakter erhält für jeden Punkt, den er in das betreffende Charakter-Attribut investiert hat, einen Bonus von +1 auf den Check.
\subsubsection{Vorteil und Nachteil}
Hat ein Charakter Vorteil bei einem Check, so würfelt er mit 2d20 und behält das höhere Ergebnis. Hat er Nachteil, so würfelt er ebenfalls mit 2d20, behält aber das niedrigere Ergebnis. Mehrere Vorteile oder Nachteile auf einen Check addieren sich nicht aufeinander. Hat ein Charakter mindesten eine Instanz von Vorteil und mindestens eine von Nachteil, so löschen sich die beiden aus und der Check wird normal durchgeführt.
\subsubsection{Check gegen Check}
Manchmal hat ein Check keine Schwierigkeitsklasse, sondern zwei oder mehr Charakter führen gegeneinander Checks aus. Beide Seiten würfeln dann 1d20 und addieren oder subtrahieren darauf jeweils ihre relevanten Boni, derjenige mit dem höheren Endergebnis gewinnt.
Bei einem Gleichstand gewinnt der Angreifer. Gibt es keinen offensichtlichen Angreifer, so verändert sich die Situation nicht
\subsubsection{Kritische Erfolge und Fehlschläge}
Führt ein Charakter einen Check durch und würfelt eine \glqq 20\grqq{}, so erzielt er einen kritischen Erfolg. Dies bedeutet, dass der Check automatisch gelingt, egal was die ursprüngliche Schwierigkeitsklasse war. Zusätzlich kann der GM, den Umständen entsprechend, einen zusätzlichen Bonus gewähren. Zeigt der Würfel bei einem Check jedoch eine \glqq 1\grqq{}, so ist dies ein kritischer Fehlschlag. Der Check schlägt automatisch fehl, und wenn ein Charakter einen Gegenstand ausgerüstet hat, so wird dieser unbrauchbar und muss mit einer Aktion wiederhergestellt werden. Kontrolliert er ein Fahrzeug, so produziert es am Beginn seines nächsten Zugs nur die Hälfte an Energie (abrunden). Zudem kann der GM den Umständen entsprechend weitere Komplikationen einführen.
\section{Pause}
Eine Pause ist ein Zeitraum von 12 Stunden, den ein Charakter in seiner Heimatbasis oder einer Einrichtung mit vergleichbaren Services verbringt. Wenn er während dieser Zeit keine anstrengenden Aktivitäten auf sich nimmt, so kann er die medizinischen Einrichtungen oder etwaige Freizeit nutzen, um alle seine Fähigkeiten zurückzuerlangen, außer er hat eine Verkrüppelung erlitten. Außerdem dient diese Zeit dazu, neue Objekte zu erwerben und nur am Ende einer Pause kann ein Charakter die Boni seines potentiellen Levelaufstieges anwenden.
\section{Erschöpfung}
Leidet ein Charakter unter Schlafentzug oder befindet sich in einer extremen Umgebung (zum Beispiel einer Wüste oder einer Rettungskapsel mit schwindender Sauerstoffversorgung), so muss er mindestens alle 6 Stunden einen Konstitutions(Widerstandsfähigkeit)-Check gegen die Schwierigkeitsklasse seiner Umgebung durchführen und bei einem Fehlschlag eine Verletzung erleiden. Solange er in dieser Umgebung ist, kann er keine Pause durchführen.

Ein Charakter kann für 3 Tage ohne ausreichende Ernährung agieren, ohne dass es seine Leistungsfähigkeit signifikant beeinträchtigt. Nach diesem Zeitraum muss er alle 6 Stunden eine Fähigkeit wählen, die verkrüppelt wird. Zudem hat er Nachteil bei allen Checks. Er kann keine Fähigkeiten auf irgendeinen Weg zurückerhalten, solange er keine passende Nahrung zu sich nimmt. Der Konsum von Nahrung in der Menge einer Ration beendet diesen Zustand. Auch nichtbiologische Charaktere müssen regelmäßig ihre Energiereserven erneuern und ihre Systeme warten oder modifizieren, und führen deshalb auch eine Form von \glqq Ration\grqq{} mit sich.
\section{Verkrüppelung}
Ein Charakter kann durch gezielte Gewalteinwirkung oder schwere Unfälle verkrüppelt werden. Wie bei einer Verletzung muss er eine Fähigkeit wegstreichen, erhält diese aber am Ende einer Pause nicht mehr zurück. Diese Verkrüppelung stellt eine schwere körperliche oder psychologische Wunde dar, von der sich ein Charakter alleine nicht mehr erholen kann.

Um eine Verkrüppelung zu beseitigen, muss ein Charakter am Beginn einer Pause ansagen, dass er die Verkrüppelung beseitigen will. Er gewinnt keine anderen Fähigkeiten zurück und würfelt am Ende der Pause 2d4. Bei einem Ergebnis von \glqq 7\grqq{} oder \glqq 8\grqq{} beendet er die Verkrüppelung und erhält die betreffende Fähigkeit zurück. Er kann sich auch alternativ für 3 Tage in intensive Behandlung (Wert: 2) begeben und am Ende dieser Zeit die Verkrüppelung beenden. Er zählt allerdings für die gesamte Zeit der Behandlung als ausgeschaltet.

Wenn ein Charakter ausgeschaltet ist und durch eine Verletzung sterben würde, kann er einen Karma-Punkt dazu benutzen, um zu überleben (siehe \secref{sec:karmapunkte}). Er wählt zwei seiner Fähigkeiten und erhält eine zufällig wieder, die andere wird verkrüppelt. Hierfür kann keine bereits verkrüppelte Fähigkeit ausgewählt werden.

\section{Tödliche Umgebung}
\section{Runden und Züge}
\section{Das Spielfeld}
\section{Fliegen und Tauchen}
\section{Kampf}
\section{Ablauf eines Zuges}
\section{Aktionen}
\section{Hacking}

\chapter{Spielregeln}
Mehrere Spieler treffen sich zu einer Session. Ein Spieler, der Gamemaster (kurz GM), leitet das Spiel an. Er beschreibt die Welt, kontrolliert die Nichtspielercharaktere (NPCs) und agiert als Schiedsrichter zwischen Spielern und Regelwerk. Die Spieler übernehmen die Rolle ihres Charakters, und beschreiben dem GM die Aktionen ihrer Charaktere, der ihnen wiederum die Folgen dieser Aktionen erklärt, und so weiter.
\textbf{Jeder am Tisch ist dafür verantwortlich, dass sich alle wohl fühlen und Spaß haben!}
\section{Checks}
Ein Check muss immer dann von einem Charakter durchgeführt werden, wenn er eine Handlung in einer schwierigen oder gefährlichen Situation ausführen will oder muss. Normalerweise entscheidet der GM, wann und was für einen Check ein Spieler durchführen muss. Um einen Check durchzuführen, würfelt ein Spieler 1d20 und addiert darauf alle Boni, die er auf den Check durch seinen Charakter, Ausrüstung, Umgebung oder andere Umstände erhält. Dieser Wert wird dann mit der Schwierigkeitsklasse des Checks verglichen. Der GM gibt vor einem Check an, um welche Art von Check es sich handelt und legt die Schwierigkeitsklasse fest. Beträgt der Wert des Checks mindestens die Höhe der Schwierigkeitsklasse, so ist er ein Erfolg.

Für Checks, die Charaktere betreffen, wird normalerweise ein Charakter-Attributs-Check durchgeführt (z.B. ein Stärke-Check). Hat ein Charakter eine Fähigkeit, von der er meint dass sie auf den Check anwendbar ist, so kann er dies dem GM mitteilen, welcher dem Spieler eventuell einen Bonus oder Vorteil auf den Check gewährt. Wird für einen Check eine spezielle Grundfähigkeit genutzt, so wird dieses hinter dem Charakter-Attribut in Klammern angegeben (z.B. ein Stärke(Nahkampf Abwehren)-Check, um von einem Nahkampfangriff nicht verletzt zu werden). Ein Charakter erhält für jeden Punkt, den er in das betreffende Charakter-Attribut investiert hat, einen Bonus von +1 auf den Check.
\subsubsection{Vorteil und Nachteil}
Hat ein Charakter Vorteil bei einem Check, so würfelt er mit 2d20 und behält das höhere Ergebnis. Hat er Nachteil, so würfelt er ebenfalls mit 2d20, behält aber das niedrigere Ergebnis. Mehrere Vorteile oder Nachteile auf einen Check addieren sich nicht aufeinander. Hat ein Charakter mindesten eine Instanz von Vorteil und mindestens eine von Nachteil, so löschen sich die beiden aus und der Check wird normal durchgeführt.
\subsubsection{Check gegen Check}
Manchmal hat ein Check keine Schwierigkeitsklasse, sondern zwei oder mehr Charakter führen gegeneinander Checks aus. Beide Seiten würfeln dann 1d20 und addieren oder subtrahieren darauf jeweils ihre relevanten Boni, derjenige mit dem höheren Endergebnis gewinnt.
Bei einem Gleichstand gewinnt der Angreifer. Gibt es keinen offensichtlichen Angreifer, so verändert sich die Situation nicht
\subsubsection{Kritische Erfolge und Fehlschläge}
Führt ein Charakter einen Check durch und würfelt eine \glqq 20\grqq{}, so erzielt er einen kritischen Erfolg. Dies bedeutet, dass der Check automatisch gelingt, egal was die ursprüngliche Schwierigkeitsklasse war. Zusätzlich kann der GM, den Umständen entsprechend, einen zusätzlichen Bonus gewähren. Zeigt der Würfel bei einem Check jedoch eine \glqq 1\grqq{}, so ist dies ein kritischer Fehlschlag. Der Check schlägt automatisch fehl, und wenn ein Charakter einen Gegenstand ausgerüstet hat, so wird dieser unbrauchbar und muss mit einer Aktion wiederhergestellt werden. Kontrolliert er ein Fahrzeug, so produziert es am Beginn seines nächsten Zugs nur die Hälfte an Energie (abrunden). Zudem kann der GM den Umständen entsprechend weitere Komplikationen einführen.
\section{Pause}
Eine Pause ist ein Zeitraum von 12 Stunden, den ein Charakter in seiner Heimatbasis oder einer Einrichtung mit vergleichbaren Services verbringt. Wenn er während dieser Zeit keine anstrengenden Aktivitäten auf sich nimmt, so kann er die medizinischen Einrichtungen oder etwaige Freizeit nutzen, um alle seine Fähigkeiten zurückzuerlangen, außer er hat eine Verkrüppelung erlitten. Außerdem dient diese Zeit dazu, neue Objekte zu erwerben und nur am Ende einer Pause kann ein Charakter die Boni seines potentiellen Levelaufstieges anwenden.
\section{Erschöpfung}
Leidet ein Charakter unter Schlafentzug oder befindet sich in einer extremen Umgebung (zum Beispiel einer Wüste oder einer Rettungskapsel mit schwindender Sauerstoffversorgung), so muss er mindestens alle 6 Stunden einen Konstitutions(Widerstandsfähigkeit)-Check gegen die Schwierigkeitsklasse seiner Umgebung durchführen und bei einem Fehlschlag eine Verletzung erleiden. Solange er in dieser Umgebung ist, kann er keine Pause durchführen.

Ein Charakter kann für 3 Tage ohne ausreichende Ernährung agieren, ohne dass es seine Leistungsfähigkeit signifikant beeinträchtigt. Nach diesem Zeitraum muss er alle 6 Stunden eine Fähigkeit wählen, die verkrüppelt wird. Zudem hat er Nachteil bei allen Checks. Er kann keine Fähigkeiten auf irgendeinen Weg zurückerhalten, solange er keine passende Nahrung zu sich nimmt. Der Konsum von Nahrung in der Menge einer Ration beendet diesen Zustand. Auch nichtbiologische Charaktere müssen regelmäßig ihre Energiereserven erneuern und ihre Systeme warten oder modifizieren, und führen deshalb auch eine Form von \glqq Ration\grqq{} mit sich.
\section{Verkrüppelung}
Ein Charakter kann durch gezielte Gewalteinwirkung oder schwere Unfälle verkrüppelt werden. Wie bei einer Verletzung muss er eine Fähigkeit wegstreichen, erhält diese aber am Ende einer Pause nicht mehr zurück. Diese Verkrüppelung stellt eine schwere körperliche oder psychologische Wunde dar, von der sich ein Charakter alleine nicht mehr erholen kann.

Um eine Verkrüppelung zu beseitigen, muss ein Charakter am Beginn einer Pause ansagen, dass er die Verkrüppelung beseitigen will. Er gewinnt keine anderen Fähigkeiten zurück und würfelt am Ende der Pause 2d4. Bei einem Ergebnis von \glqq 7\grqq{} oder \glqq 8\grqq{} beendet er die Verkrüppelung und erhält die betreffende Fähigkeit zurück. Er kann sich auch alternativ für 3 Tage in intensive Behandlung (Wert: 2) begeben und am Ende dieser Zeit die Verkrüppelung beenden. Er zählt allerdings für die gesamte Zeit der Behandlung als ausgeschaltet.

Wenn ein Charakter ausgeschaltet ist und durch eine Verletzung sterben würde, kann er einen Karma-Punkt dazu benutzen, um zu überleben (siehe \secref{sec:karmapunkte}). Er wählt zwei seiner Fähigkeiten und erhält eine zufällig wieder, die andere wird verkrüppelt. Hierfür kann keine bereits verkrüppelte Fähigkeit ausgewählt werden.

\section{Tödliche Umgebung}
Ein Charakter in einer absolut lebensgefährlichen Umgebung (wie dem Vakuum des Weltraums, einem Säurebottich oder dem aktiven Schlot eines Vulkans) erleidet solange am Ende seines Zuges eine Verletzung, bis er stirbt.
\section{Runden und Züge}
Jeder Kampf wird in Runden aufgeteilt, die jeweils einen Zeitraum von 6 Sekunden repräsentieren. In einer Runde hat jeder Teilnehmer einen Zug, während dem er handeln kann. Die Reihenfolge der Kämpfenden wird durch die Initiative bestimmt. Die Aktionen der Charaktere in einer Runde erfolgen alle, obwohl ihre Züge nacheinander ausgeführt werden, in einem Zeitraum von 6 Sekunden

Die Effekte der Systeme eines Fahrzeuges werden jeweils während des Zugs des Charakters ausgeführt, der sie kontrolliert.
\subsubsection{Initiative}
Am Beginn eines Kampfes wird ermittelt, in welcher Reihenfolge die Teilnehmer handeln können. Dazu wird von jedem ein Geschicklichkeits-Check durchgeführt und diese dann von dem höchsten zum niedrigsten Ergebnis angeordnet. In dieser Reihenfolge sind dann die Kämpfenden mit ihren Zügen an der Reihe.
\subsubsection{Überraschen}
Normalerweise bemerken sich die Teilnehmer am Beginn eines Kampfes gegenseitig, doch manchmal versucht eine Partei, die andere zu überrumpeln. Dafür muss sie sich tarnen oder anderweitig verstecken und dann aus dem Nichts angreifen. Überrascht eine Partei die andere erfolgreich, so können nur die Mitglieder der überraschenden Partei in der ersten Runde ihre Züge ausführen, die Überraschten verbringen ihre Züge damit, nichts zu tun.

Um sich einem Gegner unauffällig zu nähern, muss ein Charakter entweder einen Tarnen-, Verkleiden- oder Untertauchen-Check für das passende Attribut durchführen (für gewöhnlich entweder Geschicklichkeit oder Charisma). Um solch einen Versuch zu durchschauen, kann ein Charakter einen Check in Weisheit(Wahrnehmung) durchführen und erkennt bei einem Erfolg die wahre Natur des Charakters vor ihm.
\section{Das Spielfeld}
Charaktere und Fahrzeuge bewegen sich auf einem imaginären Spielfeld. Dieses besteht normalerweise aus einer Ebene, die mit verschiedensten Hindernissen bestückt ist.
Längenangaben werden im Regelwerk in \glqq Längen\grqq{} angegeben. Eine Länge entspricht 5 Meter innerhalb der Welt des Spieles. Sind Charaktere eine Länge voneinander entfernt oder näher, so gelten sie als direkt benachbart, sie können sich aber jeweils nicht durcheinander hindurch bewegen.

Befinden sich Spielercharaktere im Kampf mit NPCs, so wird der Konflikt meist ohne Spielfeld und nur in Erzählung ausgeführt. Wird die Situation aber zu kompliziert, so wird eine Karte als schematische Darstellung der Geschehnisse ins Spiel gebracht.

Fahrzeuge können sich auf dieser durch den Bereich eines Charakters hindurch bewegen, Spielercharaktere können sich aber nicht durch den Bereich eines Fahrzeugs oder gegnerischen NPCs bewegen. In der Regel befindet sich nur ein (lebendiger) Charakter in einem Bereich von 1 x 1 Längen.
\section{Fliegen und Tauchen}
Manche Fahrzeuge sind in der Lage, sich in einem Medium wie Luft oder Wasser dreidimensional zu bewegen. Ein Fahrzeug oder anderweitiges Objekt, dass sich außerhalb der \glqq Ebene\grqq{} befindet, wird mit einer Entsprechung dargestellt. Dazu dient der Basispunkt, der sich direkt unter- oder oberhalb des Fahrzeuges auf dem \glqq Spielfeld\grqq{} befindet. Die entsprechende Entfernung zur Ebene wird durch die sogenannte \glqq Höhe\grqq{} dargestellt, von der es drei Stufen gibt. Um ein fliegendes oder tauchendes Fahrzeug vom \glqq Spielfeld\grqq{} aus angreifen zu können, muss sich ein Charakter in einer Reichweite von 8 Längen zum Basispunkt des Ziels befinden. Zudem muss die benutzte Waffe die bei der Höhe angegebene Mindestreichweite besitzen, sonst schlägt der Angriff automatisch fehl.
\\
\begin{tabular}{c|c}
Höhe & Mindestreichweite der Waffe \\
\hline
1 & 8 Längen \\
2 & 16 Längen \\
3 & 24 Längen \\
\end{tabular}
\\
Höhen über der \glqq Spielebene\grqq{} (also Fliegen) werden positiv bezeichnet, Höhen darunter (also Tauchen) negativ. Ein Fahrzeug kann sich innerhalb seiner Höhe mit den bei seinem System angegebenen Geschwindigkeiten bewegen. Die Kosten für das Wechseln der Höhe sind ebenfalls bei dem System angegeben.

Für das Angreifen aus einer Höhe gelten die gleichen Regeln wie für gewöhnliche Fahrzeuge. Ein fliegendes oder tauchendes Fahrzeug muss seinen Basispunkt innerhalb von 8 Längen zu seinem Ziel bewegen und eine Waffe mit der zur Höhe passenden Reichweite besitzen. Zwei Fahrzeuge auf derselben Höhe nutzen die gewöhnlichen Regeln, um die Reichweite ihrer Waffen zu ermitteln und sich anzugreifen.

Ein Fahrzeug kann auf die \glqq Höhe 0\grqq{} wechseln um auf die \glqq Spielbrettebene\grqq{} zu gelangen, genauso wie es sonst die Höhe wechseln würde. Hier wird seine Fortbewegung permanent halbiert (abrunden, aber mindestens 1 Länge), dafür gelten für es die gleichen Regeln wie für jedes andere Fahrzeug auch auf dieser Ebene. Fahrzeuge können sich auch in extremeren Höhen befinden (zum Beispiel die äußeren Atmosphärenschichten oder Tiefseegraben). Dies kommt jedoch selten vor und erfordert von beiden Seiten besondere Waffen, um sich gegenseitig beeinflussen zu können.


\section{Kampf}
Im Kampf und auf Missionen kommt es ständig zu gefährlichen Situationen, durch die ein Charakter verletzt werden kann.
\subsubsection{Gefährliche Checks}
Muss ein Charakter einen gefährlichen Check ausführen, so wird er bei einem Fehlschlag verletzt und muss eine seiner Fähigkeiten wegstreichen. Gefährliche Checks sind beispielsweise Nahkampf Abwehren- und Fernkampf Ausweichen-Checks, der Versuch, ein Objekt durch ein schwelendes Feuer zu ergreifen, einem rasenden Fahrzeug auszuweichen, der Wirkung eines Giftes zu widerstehen oder in der prallen Wüstensonne zu überleben.
\subsubsection{Gleichstand bei zwei Checks}
Macht ein Charakter einen gefährlichen oder tödlichen Check gegen das Ergebnis des Checks eines anderen Charakters (zum Beispiel, um einem Fernkampfangriff auszuweichen), so gewinnen bei einem Gleichstand alle Charaktere, die einen gefährlichen oder tödlichen Check ausführen mussten.
\subsubsection{Tödliche Checks}
Muss ein Charakter einen tödlichen Check ausführen, so stirbt er bei einem Fehlschlag. Tödliche Checks sind beispielsweise der Versuch, sich ohne Rettungsleine an dem Balkon eines Hochhauses hinaufzuziehen, den Trümmern eines kollabierenden Gebäudes auszuweichen, in einen laufenden Plasma-Reaktor hineinzusehen oder eine Bombe mit bloßen Händen zu entschärfen.
\subsubsection{Ausgeschaltet}
Ist ein Charakter ausgeschaltet, so kann er sich weder eigenständig bewegen noch eine Aktion ausführen. Ein Charakter ist beispielsweise ausgeschaltet, wenn er durch Verletzungen ohnmächtig wird, er meisterhaft gefesselt ist oder durch ein Gift gelähmt wird. Ist ein Charakter ausgeschaltet, weil er alle seine Fähigkeiten wegstreichen musste, so wird jeder gefährliche Check, den er ausführen müsste, zu einem tödlichen Check. Wird ein Charakter ausgeschaltet, so erwacht er nach 1d4 Stunden wieder von selbst und kann eine Fähigkeit wählen, die er zurückerhält. Kann er aus irgendeinem Grund keine seiner Fähigkeiten wählen, so bleibt er ausgeschaltet.
\section{Ablauf eines Zuges}
Während seines Zuges kann ein Charakter sich bis zu seiner Geschwindigkeit bewegen und eine Aktion ausführen. Er kann sich auch dafür entscheiden, nichts zu tun.
\subsubsection{Bewegen}
Jeder Charakter hat eine Geschwindigkeit, die bestimmt, wie weit er sich an seinem Zug bewegen kann. Dabei kann er die Strecke seiner Bewegung beliebig auf den Zeitraum vor oder nach der Aktion verteilen.\\
\textbf{Schwieriger Untergrund:} Verschiedenste Hindernisse können das Vorankommen eines Charakters erschweren. Jede Länge, die sich ein Charakter auf schwierigem Untergrund bewegt, kostet eine zusätzliche Länge an Geschwindigkeit.\\
\textbf{Kriechen:} Während seines Zuges kann sich ein Charakter jederzeit auf den Boden werfen. Will er wieder aufstehen, so muss er 2 Längen seiner Geschwindigkeit dafür ausgeben. Solange ein Charakter kriecht, erhält er automatisch dreiviertelte-Deckung. Er hat Nachteil bei Nahkampf Abwehren-Checks. Jede Länge, die sich ein Charakter kriechend bewegt, kostet eine zusätzliche Länge an Geschwindigkeit.
\subsubsection{Aktion}
Eine Aktion ist eine Handlung des Charakters, die für gewöhnlich nicht länger als 6 Sekunden dauern sollte.
\subsubsection{Gegenstand wechseln oder aufheben}
Ein Charakter kann immer einen Gegenstand ausgerüstet haben, und während seines Zuges einmal zu einem anderen Gegenstand wechseln, ohne dafür eine Aktion zu verwenden. Stattdessen kann er auch einen Gegenstand aufheben, der weniger als eine Länge von ihm entfernt ist.
\subsubsection{Sprechen}
Während seines Zuges kann ein Charakter einige Worte oder einen kurzen Satz sprechen, ohne dafür eine Aktion verwenden zu müssen.

\section{Aktionen}
Praktisch alles, was ein Spieler außerhalb seiner Bewegung an seinem Zug macht, fällt unter die Kategorie Aktion. Eine Aktion eines Charakters stellt meist eine Interaktion mit seiner Umgebung dar,
\subsection{Angriff}
Hat ein Charakter einen Gegenstad mit der Bezeichnung \glqq Waffe\grqq{} oder vergleichbare Gegenstände ausgerüstet, so kann er einen Angriff gegen ein Ziel in seiner Reichweite ausführen. Viele Aspekte, wie die Fähigkeiten eines Charakters, die Umgebung, der Zustand des Zieles, usw. können Einfluss auf den Verlauf eines Angriffs nehmen.
\subsubsection{Nahkampfangriffe}
Damit ein Charakter einen Nahkampfangriff ausführen kann, muss sich sein Ziel in direkter Nähe (1 Länge oder weniger entfernt) befinden und er muss eine Nahkampfwaffe ausgerüstet haben. Um sein Ziel zu verletzen, kann der Charakter nun einen Stärke(Nahkampfwaffen)-Check durchführen. Ist das Ergebnis gleich oder höher als der gefährliche Stärke(Nahkampf Abwehren)-Check des Zieles, so wird es getroffen und verletzt. Wenn ein Charakter einen anderen mit einem Nahkampfangriff verletzt und folglich töten würde, so kann er ihn stattdessen ausschalten.
\subsubsection{Fernkampfangriffe}
Damit ein Charakter einen Fernkampfangriff ausführen kann, muss sich sein Ziel innerhalb der Reichweite seiner Fernkampfwaffe befinden und er muss diese Fernkampfwaffe ausgerüstet haben. Um sein Ziel zu verletzen, kann der Charakter nun einen Geschicklichkeit(Fernkampfwaffen)-Check durchführen. Ist das Ergebnis gleich oder höher als der gefährliche Geschicklichkeit(Fernkampf Ausweichen)-Check des Zieles, so wird es getroffen und verletzt. Führt ein Charakter einen Fernkampfangriff (mit einer Waffe, deren Reichweite größer als 1 Länge ist) gegen ein Ziel aus, dass 1 Länge oder weniger von ihm entfernt ist, so hat er Nachteil bei seinem Geschicklichkeit(Fernkampfwaffen)-Check.
\subsubsection{Starke Angriffe}
Fügt eine Attacke (zum Beispiel die einer Fahrzeug- oder schweren Waffe) einem Charakter in einem Angriff mehrere Verletzungen zu, so muss er so viele Fähigkeiten wegstreichen, wie er Punkte an Schaden erleiden würde. Wird er durch den Teilschaden eines Angriffs ausgeschaltet und es bleibt noch eine Anzahl an Verletzungen übrig, so verfallen sie für diese eine Attacke. Jede weitere Verletzung, die ein ausgeschalteter Charakter durch weitere Angriffe erhält, verläuft tödlich.
\subsection{Verteidigung}
\subsubsection{Nahkampf Abwehren}
Würde ein Charakter von einem Nahkampfangriff getroffen und hat einen Gegenstand mit der Eigenschaft \glqq Waffe\grqq{} ausgerüstet, so kann er einen gefährlichen Stärke(Nahkampf Abwehren)-Check durchführen. Ist das Ergebnis höher als der Stärke(Nahkampfwaffen)-Check des Nahkampfangriffs, so verletzt die Attacke den Charakter nicht. Hat ein Charakter keine Waffe ausgerüstet oder ist ausgeschaltet, so beträgt der Grundwert seines Stärke(Nahkampf Abwehren)-Checks automatisch 5. Jeder zusätzliche Bonus eines Charakters zu Nahkampf Abwehren und seine Stärke werden addiert.
\subsubsection{Fernkampf Ausweichen}
Würde ein Charakter von einem Fernkampfangriff getroffen, so kann er einen gefährlichen Geschicklichkeit(Fernkampf Ausweichen)-Check durchführen. Ist das Ergebnis höher als der Geschicklichkeit(Fernkampfwaffen)-Check des Fernkampfangriffes, so verletzt die Attacke den Charakter nicht. Kann ein Charakter seinen Angreifer nicht sehen, ist sich dessen Präsenz nicht bewusst oder ist ausgeschaltet, so beträgt der Grundwert seines Geschicklichkeit(Fernkampf Ausweichen)-Checks automatisch 5. Jeder zusätzliche Bonus eines Charakters zu Fernkampf Ausweichen, seine Geschicklichkeit und eventuelle Deckung werden addiert.
\subsubsection{Deckung}
Befindet sich ein Charakter in Deckung, so bekommt er einen Bonus auf seine Fernkampf Ausweichen-Checks, solange sich sein Angreifer auf der anderen Seite der Deckung befindet (Halbe Deckung: +2, Dreiviertelte Deckung: +5). Ein Charakter in voller Deckung kann nicht direkt von Fernkampfangriffen getroffen werden.
\subsubsection{Ausweichen}
Ein Charakter kann seine Aktion dafür verwenden, um auf alle Nahkampf Abwehren- und Fernkampf Ausweichen-Checks bis zum Beginn seines nächsten Zuges einen Bonus von +5 zu erhalten.
\subsection{Sonstige Kampfhandlungen}
\subsubsection{Sprinten}
Ein Charakter kann seine Aktion dazu verwenden, um zu sprinten. Er kann sich um 4 weitere Längen bewegen.
\subsubsection{Zielen}
Ein Charakter kann seine Aktion dafür verwenden, um mit seiner Waffe zu zielen. Dafür muss er einen Charakter, den er sehen kann, als sein Ziel festlegen. Er erhält einen Bonus auf den ersten Check, den er während seines nächsten Zuges ausführt, um sein Ziel mit einer Waffe zu verletzen. Dieser Bonus verfällt, wenn der Charakter an seinem Zug eine andere (oder keine) Aktion ausführt, sich sein Ziel in volle Deckung begibt oder er es auf andere Weise aus den Augen verliert.

Zielt ein Charakter mehrere Runden hintereinander, erhöht er die Stufe seines Zielens jede Runde um 1, bis zu einem maximum von 3. Jede Stufe beinhaltet auch die Boni aller vorherigen Stufen.
\begin{tabular}{c|c}
Stufe & Bonus\\		
\hline
1 & +2 auf den Check zum Angriff \\
2 & Ignoriere Deckungs-Bonus \\
3 & Verdopple Reichweite der Waffe \\
\end{tabular}
\subsubsection{Helfen}
Ein Charakter kann seine Aktion dafür verwenden, einem Verbündeten bei einem Vorhaben zu helfen. Bis zu Beginn des nächsten Zuges des Charakters hat der Verbündete einmal Vorteil auf einen Check, bei dem es Sinn macht, dass Kollaboration hilfreich ist. Befindest du dich in direkter Nähe zu einem Gegner, so kannst du einem Verbündeten mit der Helfen-Aktion bei einem Nahkampf- oder Fernkampfangriff gegen diesen Gegner bis zum Beginn deines nächsten Zuges Vorteil verschaffen.
\subsubsection{Vorbereiten}
Wenn ein Charakter eine Aktion vorbereitet, so legt er während seines Zuges den Auslöser für die vorbereitete Aktion und ihre genaue Natur fest. Sollten bis zum Beginn seines nächsten Zuges die Kriterien des Auslösers erfüllt werden, so kann der Charakter sofort nach der Beendigung des Auslösers seine Aktion auch außerhalb seines Zuges durchführen. Auf diese Weise kann von einem Charakter immer nur eine Aktion pro Runde ausgelöst werden.
\subsection{Allgemeine Aktionen}
\subsubsection{Objekt benutzen}
Ein Charakter kann seine Aktion dazu benutzen, mit einem Objekt in seinem Inventar oder in seiner unmittelbaren Umgebung zu interagieren.
\subsubsection{Fahrzeug oder System kontrollieren}
Will ein Charakter ein Fahrzeug kontrollieren, so muss er dafür jeden Zug seine Aktion dafür verwenden und sich an den Kontrollen befinden. Das Betreten und Verlassen eines Fahrzeugs kostet jeweils eine Aktion.
\subsubsection{Andere Aktionen}
Es können noch eine Vielzahl an anderen Aktionen ausgeführt werden. Es sollte sich dabei normalerweise um eine Aktivität handeln, die nicht länger als 6 Sekunden andauert.
\section{Hacking}
Ein Charakter kann versuchen in ein Computersystem einzudringen. Hierfür benötigt er eine Hacking-Konsole, ein Deck, einen Computer in seinem Inventar oder Zugang zu einer vergleichbaren Einrichtung. Zum Hacken kann er dann einen Intelligenz(Software)-Check durchführen, gegen den sich das System mit einem Intelligenz(Software)-Check verteidigen kann. Der Einfluss, den ein Charakter beim Hacking nehmen kann, wird durch die Sicherheitsstufen des Computersystems bestimmt. Jede Stufe gibt dem System einen Bonus auf den Intelligenz(Software)-Check zur Verteidigung, der der Höhe der Sicherheitsstufe entspricht. Solange der Hack andauert, muss ein Charakter jeden Zug seine Aktion darauf verwenden. Sobald er sie für einen Zug nicht anwendet, wird der Hack beendet.

Zudem muss noch eine weitere Voraussetzung an beiden Seiten erfüllt sein. Ein Charakter muss den Check, auch wenn er beim ersten Mal gelingt, in regelmäßigen Abständen wiederholen. Diese sind bei der Stufe des Hacks als \glqq Dauer\grqq{} angegeben. Ein System sucht automatisch regelmäßig nach Eindringlingen und Sicherheitslücken, es bemerkt den Charakter also am Anfang seines Zuges, wenn er diesen wiederholten Check nicht schafft.

In einer Runde (also 6 Sekunden), kann ein Charakter an seinen Zug mit 2 Teilen des Systems interagieren. Eine Interaktion ist zum Beispiel das Aufrufen, Manipulieren, Kopieren oder Löschen einer Datei oder das übernehmen eines Stücks Hardware für die Dauer des Hacks oder das Aktivieren eines Programms.

Ein Charakter kann jederzeit während seines Zuges den Hack beenden. Alle Änderungen, die er erfolgreich durchgeführt hat, bleiben bestehen (und wenn sein Eindringen nicht bemerkt wurde, wird das System auch nicht besonders auf Manipulation überprüft werden).
\begin{tabular}{c|c|c|p{3cm}}
Stufe & Voraussetzung & Dauer & Effekt \\		
\hline
+0 & Internetverbindung	& permanent & Öffentlich freigegebene Informationen; offizielle Dienstleistungen \\
+1 & Internetverbindung	& 1 Stunde	& Hintergrundinformationen über Struktur und Abläufe des Systems und Aktivitäten darin \\
+2 & Physischer Link & 1 Minute & Manipulation einfacher Informationen; Ausführen einfacher Aktionen; Steuern von simpler Hardware \\
+3 & Physischer Link & 1 Runde & Zugriff auf und Manipulation von gesicherter Information; Ausführen aller Aktionen, die von Computersystem übernommen werden; Steuern von systemweiter Hardware \\
\end{tabular}
\todo{Tabelle anpassen}
\subsection{Clips}
Ein Clip dient zur Verlängerung der Reichweite von Hackern. Die höheren Funktionen moderner Computersysteme können nicht mehr aus Entfernung gehackt werden, sondern es muss eine physische Verbindung mit einem Teil des Systems geschlossen werden. Mit einem Clip kann genau das getan werden, ohne dass sich ein Hacker eigenständig in Gefahr bringen muss. Sobald dieser unauffällige, streichholzschachtelgroße Kasten physisch an einer Systemschnittstelle angebracht wird, verbindet er sich sowohl mit dem System, als auch drahtlos mit dem kontrollierenden Hacker. Dieser kann dann in das System eindringen, ohne seine Operationsbasis verlassen zu müssen. Das einzige Problem ist, dass ein verbündeter Infiltrationsspezialist den Clip vorher im Zielsystem installiert haben (für die Installation ist keine besondere Einweisung nötig) oder der Clip mit einem Computerteil hineingeschmuggelt werden muss.

Checks zum Hacking werden mit einem Chip ganz normal durchgeführt, schlägt aber ein Check fehl, so verliert der Clip einen seiner 3 Trefferpunkte, egal ob der Hack zurückverfolgt wurde oder nicht. Hat der Clip alle Trefferpunkte verloren, so wird er unbrauchbar.
Ein Clip bezieht seine Energie direkt aus seinem Zielsystem und kann (je nach Häufigkeit von physischen Kontrollen durch die Besitzer des Systems) theoretisch für Jahre unentdeckt bleiben. Einen Clip mit sich zu führen ist höchst illegal.
\subsection{Fehlschlag}
Schlägt der Hack fehl, so wird das System alarmiert. Ein alarmiertes System hat Vorteil bei Intelligenz(Software)-Checks zur Verteidigung gegen Hacker für 1d4 Stunden. Zudem können auch noch andere Komplikationen auftreten, wie ein plötzliches Isolieren oder Herunterfahren des Computersystems oder das Erscheinen von Sicherheitskräften.

Solange das alarmierte System mindestens eine Firewall hat, kann es dem Hacker noch zusätzliche Schwierigkeiten bereiten: Das System würfelt 1d4 + Sicherheitsstufe + Anzahl der Firewalls. Ist das Ergebnis größer als 6, so kann das System die Position des Hackers zurückverfolgen.
\subsection{Computersystem verteidigen}
Ein Charakter mit der Grundfähigkeit \glqq Software\grqq{} kann jeden Zug seine Aktion darauf verwenden, ein Computersystem gegen Hacking zu verteidigen. Wird das System angegriffen, so macht er Stellvertretend für das System den Intelligenz(Software)-Check. Zudem bemerkt er automatisch den Angriff, wenn dieser fehlschlägt. Ein System, das nicht aktiv verteidigt wird, hat ein Intelligenz-Attribut von +0 und im unalarmierten Zustand keinen Vorteil auf den Check.

Im System eingebaute Firewalls können das Hacking stark erschweren, da sie dem System bei der Verteidigung einen Bonus von +5 geben. Zudem kann der momentane Standort des Hackers zurückverfolgt werden, wenn ihm der Hack misslingt und das Computersystem mindestens eine Firewall besitzt. Deshalb ist es ratsam, Firewalls vor dem Hack zu zerstören oder mit Narco-Tec auszuschalten. 